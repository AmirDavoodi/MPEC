\documentclass{article}
\usepackage{amsmath}
\usepackage{amssymb}
\title{Introduction to Addition via Recursion}
\author{}
\date{}

\begin{document}
\maketitle

\section{Recursive Definition}
For non-negative integers \(a\) and \(b\):

\[
    a + b = \begin{cases}
        a                 & \text{if } b = 0 \quad \text{(Base Case)}      \\
        (a + (b - 1)) + 1 & \text{if } b > 0 \quad \text{(Recursive Case)}
    \end{cases}
\]

\section{Expanded Recursion Steps}
The recursive case systematically reduces any addition problem to successive simpler cases through these steps:

\begin{enumerate}
    \item \textbf{Initial Problem}: \(a + b\) where \(b > 0\)

    \item \textbf{Recursive Decomposition}:
          \[
              a + b = (a + \underbrace{(b - 1)}_{\text{Simpler term}}) + 1
          \]
          This creates:
          \begin{itemize}
              \item A simpler subproblem: \(a + (b - 1)\)
              \item A pending operation: \(+ 1\)
          \end{itemize}

    \item \textbf{Iterative Reduction}:
          Repeat until reaching base case:
          \[
              \begin{aligned}
                   & a + b                              \\
                   & \Downarrow                         \\
                   & (a + (b-1)) + 1                    \\
                   & \Downarrow                         \\
                   & ((a + (b-2)) + 1) + 1              \\
                   & \Downarrow                         \\
                   & \quad \vdots                       \\
                   & \Downarrow                         \\
                   & (\cdots((a + 0) + 1) + \cdots + 1) \\
              \end{aligned}
          \]

    \item \textbf{Base Case Resolution}:
          When \(b - n = 0\):
          \[
              \underbrace{(\cdots((a + 0)}_{\text{Base case}} \underbrace{+ 1) + \cdots + 1)}_{b \text{ times}}
          \]

    \item \textbf{Result Construction}:
          \[
              a + \underbrace{1 + 1 + \cdots + 1}_{b \text{ times}} = a + b
          \]
\end{enumerate}

\section{Example}
Complete Recursion Example, for \(3 + 2\):

\[
    \begin{aligned}
        3 + 2 & = (3 + 1) + 1 \quad       & \text{(First decomposition)}  \\
              & = ((3 + 0) + 1) + 1 \quad & \text{(Second decomposition)} \\
              & = (3 + 1) + 1 \quad       & \text{(Base case applied)}    \\
              & = 4 + 1 \quad             & \text{(First increment)}      \\
              & = 5 \quad                 & \text{(Final result)}
    \end{aligned}
\]

\section*{Recursion Pattern}
General form for \(a + b\):

\[
    \begin{aligned}
        a + b & = (a + (b-1)) + 1                                       \\
              & = ((a + (b-2)) + 1) + 1                                 \\
              & \;\;\vdots                                              \\
              & = (\cdots((a + 0) + 1) + \cdots + 1)                    \\
              & = a + \underbrace{1 + 1 + \cdots + 1}_{b \text{ times}} \\
              & = a + b
    \end{aligned}
\]

\end{document}