\documentclass{article}
\usepackage{amsmath}
\usepackage{tikz}
\usepackage{enumitem}
\begin{document}

\title{Tutorial: Adding Two-Digit Numbers with Carrying}
\author{}
\date{}
\maketitle

\section*{Problem Statement}
What is $47 + 58$?

\section*{Step-by-Step Solution}

We will use the column method and carry over when necessary.

\begin{enumerate}[label=Step \arabic*:, leftmargin=*]
    \item \textbf{Write the numbers in column format.}
          \[
              \begin{array}{r}
                  47  \\
                  +58 \\
                  \hline
              \end{array}
          \]

    \item \textbf{Add the ones digits:} $7 + 8 = 15$.

          \begin{itemize}
              \item Write down the \textbf{5} in the ones place.
              \item Carry over the \textbf{1} to the tens place.
          \end{itemize}

    \item \textbf{Add the tens digits:} $4 + 5 = 9$.

          \begin{itemize}
              \item Add the carried over $1$: $9 + 1 = 10$.
              \item Write down the \textbf{0} in the tens place and carry over \textbf{1} to the hundreds.
          \end{itemize}

    \item \textbf{Write the final result:}

          \begin{itemize}
              \item The hundreds place is just the carry-over $1$.
              \item So, the final answer is:
                    \[
                        \begin{array}{r}
                            47  \\
                            +58 \\
                            \hline
                            105
                        \end{array}
                    \]
          \end{itemize}
\end{enumerate}

\section*{Conclusion}
We used the method of addition with carrying to solve $47 + 58 = 105$.

\end{document}
