\documentclass{article}
\usepackage{amsmath}
\usepackage{amssymb}
\title{Introduction to Multiplication via Recursion}
\author{}
\date{}

\begin{document}
\maketitle

\section{Definition}
Multiplication of two non-negative integers \( a \) and \( b \), denoted \( a \times b \) or \( a \cdot b \), can be defined recursively using addition:

\begin{itemize}
    \item \textbf{Base Case}: If \( b = 0 \), then \( a \times b = 0 \).
    \item \textbf{Recursive Case}: If \( b > 0 \), then \( a \times b = a + (a \times (b - 1)) \).
\end{itemize}

This definition reduces multiplication to repeated addition. Each recursive step decrements \( b \) until it reaches the base case \( b = 0 \).

\section{Examples}

\subsection{Example 1: \( 3 \times 0 \)}
Applying the base case directly:
\[
3 \times 0 = 0
\]

\subsection{Example 2: \( 0 \times 4 \)}
Using the recursive definition:
\[
\begin{aligned}
0 \times 4 &= 0 + (0 \times 3) \\
&= 0 + (0 + (0 \times 2)) \\
&= 0 + (0 + (0 + (0 \times 1))) \\
&= 0 + (0 + (0 + (0 + (0 \times 0)))) \\
&= 0 + (0 + (0 + (0 + 0))) \\
&= 0 + (0 + (0 + 0)) \\
&= 0 + (0 + 0) \\
&= 0 + 0 \\
&= 0
\end{aligned}
\]

\subsection{Example 3: \( 2 \times 3 \)}
Breaking down the recursion step-by-step:
\[
\begin{aligned}
2 \times 3 &= 2 + (2 \times 2) \\
&= 2 + (2 + (2 \times 1)) \\
&= 2 + (2 + (2 + (2 \times 0))) \\
&= 2 + (2 + (2 + 0)) \\
&= 2 + (2 + 2) \\
&= 2 + 4 \\
&= 6
\end{aligned}
\]

\subsection{Example 4: \( 5 \times 2 \)}
Applying the recursive steps:
\[
\begin{aligned}
5 \times 2 &= 5 + (5 \times 1) \\
&= 5 + (5 + (5 \times 0)) \\
&= 5 + (5 + 0) \\
&= 5 + 5 \\
&= 10
\end{aligned}
\]

\subsection{Example 5: \( 1 \times 5 \)}
Demonstrating recursion with identity multiplication:
\[
\begin{aligned}
1 \times 5 &= 1 + (1 \times 4) \\
&= 1 + (1 + (1 \times 3)) \\
&= 1 + (1 + (1 + (1 \times 2))) \\
&= 1 + (1 + (1 + (1 + (1 \times 1)))) \\
&= 1 + (1 + (1 + (1 + (1 + (1 \times 0))))) \\
&= 1 + (1 + (1 + (1 + (1 + 0)))) \\
&= 1 + (1 + (1 + (1 + 1))) \\
&= 1 + (1 + (1 + 2)) \\
&= 1 + (1 + 3) \\
&= 1 + 4 \\
&= 5
\end{aligned}
\]

\section{Conclusion}
This recursive framework demonstrates how multiplication is equivalent to repeated addition. By systematically reducing \( b \) and leveraging the base case \( b = 0 \), the definition breaks down complex operations into simpler, foundational steps. Recursion provides a clear algorithmic structure for understanding multiplication.

\end{document}