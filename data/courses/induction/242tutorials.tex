\documentclass[12pt, reqno]{article}

\usepackage{amsmath, amssymb, amsthm}
\usepackage{hyperref}
\usepackage{enumitem}

\numberwithin{equation}{section}

\title{Analysis 1 Tutorial Notes}
\author{Dana Berman}
\date{Fall 2018}

\newtheorem{prop}{Problem}[section]
\newtheorem{lem}[prop]{Lemma}
\newtheorem{thm}{Theorem}[section]
\newtheorem{cor}[thm]{Corollary}

\theoremstyle{definition}
\newtheorem{def_}{Definition}

\theoremstyle{remark}
\newtheorem{rem}{Remark}

\newcommand{\NN}{\mathbb{N}}
\newcommand{\QQ}{\mathbb{Q}}
\newcommand{\RR}{\mathbb{R}}
\newcommand{\ZZ}{\mathbb{Z}}
\newcommand{\set}[1]{\left\{#1\right\}}
\newcommand{\abs}[1]{\left\lvert#1\right\rvert}

\renewcommand{\P}{\mathcal{P}}
\renewcommand{\epsilon}{\varepsilon}

\begin{document}
\maketitle

This is a compilation of the tutorial notes I typed up while I was the TA for Math 242 (Analysis 1) at McGill university.

\section*{Tutorial 1}

\begin{enumerate}[leftmargin=*]
	\item Write down the following statement using quantifiers:

	      \emph{For any positive number $a$ we can find a negative number $b$ so that $a+b = 0$}
	      \[
		      \forall a > 0, \, \exists b < 0 \text{ such that } a+b = 0
	      \]
	\item A number is said to be rational if it can be written as a quotient of integers. For instance, the $\pi, \sqrt{2}$ are not rational while $0.\overline{33} = \frac{1}{3}$ is rational. How would you define a rational number using quantifiers? Note that the set of integers is denoted $\ZZ$.

	      A number $x$ is said to be rational if
	      \[
		      \exists p, q \in \ZZ \text{ with } q\neq 0 \text{ such that } x = \frac{p}{q}
	      \]
	      The set of rational numbers is therefore
	      \[
		      \QQ = \set{\frac{p}{q} \mid p, q \in \ZZ,\, q \neq 0}
	      \]
	\item State the negation of the following statement and declare whether it is true or false.
	      \[
		      \forall a \in \NN \quad\exists b \in \QQ \quad ab = 1
	      \]
	      Negation:
	      \[
		      \exists a\in \NN \quad \forall b \in \QQ \quad ab \neq 1
	      \]
	      The statement is true.
	\item State the negation of the following statement and declare whether it is true or false.
	      \[
		      \forall a \in \ZZ \quad a< 0 \text{ or } a>0
	      \]
	      Negation:
	      \[
		      \exists a \in \ZZ \quad a \geq 0 \text{ and } a\leq 0
	      \]
	      The statement is false.
	\item Write a statement which is logically equivalent to ``$x=2$ or $x=-2$''.

	      There are many possible solutions to this, but one example would be;
	      \[
		      x^2 = 4
	      \]
	\item Show that the sum of two odd numbers is even
	      \begin{proof}
		      Let $a,b$ be given odd numbers. Then there exists $x,y\in \ZZ$ such that $a = 2x+1$ and $b = 2y+1$. Ergo,
		      \[
			      a+b = 2x+1+2y+1 = 2x+2y+2 = 2(x+y+1)
		      \]
		      which shows that $a+b$ is even.
	      \end{proof}

	\item Given arbitrary sets $A$ and $B$, show that
	      \[
		      (A/B)\cup (B/A) = (A\cup B) \setminus (A\cap B)
	      \]
	      The above is known as the symmetric difference of $A$ and $B$ and is denoted $A\triangle B$.

	\item When proving an implication, one can instead prove it's contrapositive. Suppose I wanted to prove that \emph{if $x^2$ is even then $x$ is even} by contrapositive, what statement would I be proving.

	      I would prove that if $x$ is not even then $x^2$ is not even. Put otherwise, if $x$ is odd then $x^2$ is also odd. So, the statement \emph{if $x^2$ is even then $x$ is even} is logically equivalent to \emph{if $x$ is odd then $x^2$ is odd}.

	\item What is the contrapositive of \emph{if $n\nmid ab$ then $n\nmid a$ and $n\nmid b$} where $a,b,n \in \NN$?

	      It is \emph{if $n\mid a$ or $n\mid b$ then $n\mid ab$}.

	\item Prove by contrapositive that for every $n\in \NN$, if $n^2 - 1$ is not divisible by 8 then $n$ is even.

	      The contrapositive is: If $n$ is odd then $n^2 - 1$ is divisible by 8.
	      We will now prove this statement. We will begin with the assumption that $n$ is odd and show that $n^2-1$ is divisible by 8. When writing a proof, we are careful to make sure that every step follows. If a proof is not written in the right order, it may technically be incorrect.
	      \begin{proof}
		      We are assuming that $n$ is odd. This means that there exists a non negative integer $k$ such that
		      \[
			      n = 2k+1
		      \]
		      We now compute $n^2-1$ but substituting $n$ for $2k+1$.
		      \begin{align*}
			      n^2 - 1 & = \left(2k+1\right)^2 - 1               \\
			              & =\left(2k\right)^2 + 2(2k)(1) + 1^2 - 1 \\
			              & =4k^2 + 4k
		      \end{align*}
		      From the above it is clear that $n^2-1$ is divisible by 4 but it is still not clear that it is divisible by 8. Since we are interested in the factors of $n^2-1$, we will try factoring our equation to see if that provides us with any further information.

		      We have
		      \[
			      n^2 - 1 = 4k^2 + 4k = 4(k)(k+1)
		      \]
		      Since one of $k$ or $k+1$ is even, we see that $n^2 -1$ will be divisible by 8. We rigorously write out this thought process.

		      \textbf{Case 1}: The number $k$ is even. In this case, there exists $\ell \in \ZZ$ such that $k  = 2\ell$. Then
		      \[
			      n^2 -1= 4(k)(k+1) = 4(2\ell)(2\ell+1) = 8(\ell)(2\ell+1)
		      \]
		      Thus, $n^2-1$ is clearly divisible by 8.

		      \textbf{Case 2}: The number $k$ is odd. (Would someone like to try this on the board?) In this case, there exists $\ell \in \ZZ$ such that $k  = 2\ell + 1$. Then
		      \[
			      n^2 -1= 4(k)(k+1) = 4(2\ell + 1)(2\ell+2) = 8(2\ell + 1)(\ell+1)
		      \]
		      Thus, $n^2-1$ is clearly divisible by 8.
	      \end{proof}

	\item\label{evensquare} Recall earlier when we discussed the contrapositive of the statement \emph{if $x^2$ is even then $x$ is even}? Would someone like to come prove it by contrapositive? That is, show that \emph{if $x$ is odd then $x^2$ is odd}?
	      \begin{proof}
		      If $x$ is odd then $x = 2k + 1$ for some $k\in \ZZ$. Thus,
		      \[
			      x^2 = (2k+1)^2 = 4k^2 + 4k + 1 = 2(2k^2+2k) + 1
		      \]
		      Ergo, $x^2$ is odd.
	      \end{proof}
	\item Show by contrapositive that \emph{if $n\nmid ab$ then $n\nmid a$ and $n\nmid b$}.

	      \begin{proof}
		      We will prove that \emph{if $n\mid a$ or $n\mid b$ then $n\mid ab$}. Without loss of generality, we may suppose that $n \mid a$ (otherwise we just relabel $a$ as $b$ or $b$ as $a$, is this clear?).

		      Then there exists $x \in \ZZ$ such that $a = nx$. Thus,
		      \[
			      ab = nxb
		      \]
		      and it is clear that $n\mid ab$.
	      \end{proof}

	\item Later on in this course you will be proving the existence of the square root of 2. That is, that there exists a real number denoted $\sqrt{2}$ such that the square of that number is precisely 2. You can find numbers that whose square is very close to 2, but the existence of a precise square root is something you will prove.

	      Since there is a lot of material that you haven't covered yet, we cannot prove it's existence right now. So, we will assume that the square root of 2 exists and prove that it is irrational.

	      \begin{proof}
		      We will proceed by contradiction. This means that we take the assumptions of the statement and ask: ``what if the conclusion is false?'' From there we will derive a contradiction. This shows that the conclusion cannot be false, i.e. must be true.

		      So, we assume that the square root of 2 is rational. That is, suppose there exists $p, q \in \ZZ$ such that
		      \[
			      \sqrt{2} = \frac{p}{q}
		      \]
		      After dividing out the common factors, we may suppose that $p$ and $q$ are relatively prime.

		      Now, squaring either side of our equation we see that
		      \[
			      2 = \frac{p^2}{q^2} \implies 2q^2 = p^2
		      \]
		      In particular, $p^2$ is even. By problem \eqref{evensquare} (where we showed that if $x^2$ is even then $x$ is even) we know that $p$ is even. Thus, there exists $a \in \ZZ$ such that $p = 2a$. Substituting $p$ for $2a$ we have
		      \[
			      2q^2 = p^2 = (2a)^2 = 4a^2 \implies q^2 = 2a^2
		      \]
		      But then, $q^2$ is also even which implies that $q$ is even. So we see that both $p$ and $q$ are even. But then $p$ and $q$ are not relatively prime which is a contradiction.

		      This shows that the square root of 2 is indeed irrational.
	      \end{proof}

	\item Next course, you will cover what is known as the \emph{principle of mathematical induction}. While I will not prove this here nor state the principle formally, I want to give you guys an example of how this principle can be used.

	      The idea is that we prove a statement in what is called the \emph{base case}. Then we show that if the statement holds for $n$, then it holds for $n+1$ (\emph{induction hypothesis}). For instance, suppose we want to prove that a statement $P(n)$ is true for all $n\in \NN$. Then we prove that it is true for $n=1$, then we prove that if the statement holds for some $n$, then it also holds for $n+1$. This will show that the statement is true for all $n\in \NN$. Why? Because we have shown that the statement is true for $n=1$, thus it is also true for $1+1=2$. But then it is also true for $2+1=3$ and so on. So we conclude that the statement is true for all $n\in \NN$.

	      For example, show that for all $n\in \NN$, it holds that
	      \[
		      \sum_{k=1}^n k := 1+2+\dots + n = \frac{n(n+1)}{2}
	      \]
	      \begin{proof}
		      So, clearly the statement holds for $n=1$;
		      \[
			      1 = \frac{1(2)}{2} = \frac{1(1+1)}{2}
		      \]
		      This proves the base case. We now do the induction step. Here, we suppose the statement holds for some $n\geq 1$. Then for $n+1$ we have
		      \begin{align*}
			      \sum_{k=1}^{n+1} k & := \underbrace{1+2+\dots + n}_{= n(n+1)/2 \text{ by the induction hypothesis}} + (n+1) \\
			                         & = \frac{n(n+1)}{2} + (n+1)                                                             \\
			                         & = (n+1)\left(\frac{n}{2} + 1\right)                                                    \\
			                         & = (n+1)\left(\frac{n+2}{2}\right)                                                      \\
			                         & =\frac{(n+1)([n+1]+1)}{2}
		      \end{align*}
		      This shows that the statement also holds for $n+1$ which concludes the proof.
	      \end{proof}
	\item Show that for all integers $n \geq 4$ one has $2^n < n!$.

	      \begin{proof}
		      We proceed by induction on $n \geq 4$. Since we want to prove a statement for $n \geq 4$, our base case should be verifying the case $n = 4$.\\

		      \noindent\underbar{Base case}. For $n=4$ we manually verify the claim:
		      \[
			      2^4 = 16 < 24 = 4!.
		      \]
		      \underbar{Inductive step}. Now, we make the induction hypothesis. Suppose that, for $n \geq 4$, we know
		      \begin{equation}\label{eq:1-5}
			      2^n < n!.
		      \end{equation}
		      We want to show that the statement also holds for $n+1$. More precisely, we want to prove the following inequality: \[2^{n+1} < (n+1)!.\] By the inductive hypothesis \eqref{eq:1-5}, we have
		      \[
			      2^{n+1} = 2\cdot 2^n < 2\cdot n!
		      \]
		      Because $n \geq 4$, we also get $(n+1) \geq 5 \geq 2$. Using this together with the equation above yields
		      \[
			      2^{n+1} < 2\cdot n! < (n+1)\cdot n! = (n+1)!
		      \]
		      This completes the inductive step.
	      \end{proof}
	\item Show that $5^n - 1$ is divisible by $4$ for all $n \in \NN$.
	      \begin{proof}
		      Since we wish to prove a claim for all $n \geq 1$, we should take $n := 1$ as a base case. \\

		      \noindent\underbar{Base case}. Clearly,
		      $$5^1 - 1 = 4,$$  which is certainly divisible by $4$. Hence, the base case is true.\\

		      \noindent\underbar{Inductive step}. Assume that $5^n -1$ is divisible by $4$ for some $n \geq 1$ (this is the inductive hypothesis). We must show that $5^{n+1} - 1$ is also divisible by $4$. But, this is easily seen by writing
		      \begin{align*}
			      5^{n+1} - 1 = 5 \cdot 5^n - 1 & = \underbrace{5^n + 5^n + 5^n + 5^n}_{\text{4 times}} + \left(5^n - 1\right) \\
			                                    & = 4 \cdot 5^n + (5^n-1).
		      \end{align*}
		      Clearly, $4 \cdot 5^n$ is divisible by $4$. By the induction hypothesis, so is $5^n - 1$. Hence, $4$ must be a divisor of their sum, which is equal to $5^{n+1} -1$ by the calculations above.
	      \end{proof}


	\item Show that $2^n \geq 2n+1$ for all integers $n \geq 3$.

	      \begin{proof}
		      This looks like something we should try to prove with induction.\\

		      \noindent\underbar{Base case}. With $n= 3$, we check directly that
		      \[
			      2^3 = 8> 7 = 2(3)+1
		      \]
		      This means that the base case is true.\\

		      \noindent\underbar{Inductive step}. Suppose that the statement is true for some $n \geq 3$; we must prove that it is also true for $n+1$. Namely, our induction hypothesis is:
		      \begin{equation}\label{eq:1-0}
			      2^n \geq 2n+1.
		      \end{equation}
		      We want to show that $2^{n+1}\geq 2(n+1) + 1 $. First, write
		      \begin{align}\label{eq:1-00}
			      2^{n+1} = 2\cdot 2^n \overset{\eqref{eq:1-0}}{\geq} 2\cdot (2n+1)
		      \end{align}
		      Since $n \geq 3$, it is clear that $2n \geq 1$. So we have
		      \[
			      2^{n+1} \geq 2\cdot (2n+1) = 2(n+1) + 2n \geq 2(n+1) + 1
		      \]
		      This completes the inductive step and the claim is proven.
	      \end{proof}

	\item Show that if $n \geq 5$ is an integer, then $2^n > n^2$.
	      \begin{proof}
		      Since we are asked to prove an equality involving integers, it is a good idea to try and use mathematical induction.  We start with the base case $n=5$.\\

		      \noindent\underbar{Base case}. With $n= 5$ we check that
		      \[
			      2^5 = 32 > 25 = 5^2
		      \]
		      Hence, the claim holds for $n = 5$.\\

		      \noindent\underbar{Inductive step}. Suppose the claim is true for $n \geq 5$, we must show that is is also true for $n+1$. Our inductive hypothesis is therefore the following:
		      \[
			      2^n > n^2
		      \]
		      Using this, we must show that
		      \[
			      2^{n+1} > (n+1)^2
		      \]
		      Using the induction hypothesis, we calculate
		      \[
			      2^{n+1} = 2^n + 2^n > n^2 + 2^n
		      \]
		      Because $n \geq 5 \geq 3$, we can apply the previous problem to the above. Doing so gives $2^n > 2n+1$ whence
		      \[
			      2^{n+1} > n^2 + 2^n > n^2 + 2n + 1 = (n+1)^2.
		      \]
		      This completes the inductive step.
	      \end{proof}

\end{enumerate}

\section*{Tutorial 2}

\begin{enumerate}[leftmargin=*]
	\item Conjecture and prove a formula for the sum
	      \[
		      \frac{1}{1\cdot 3} +\frac{1}{3\cdot 5} + \frac{1}{5\cdot 7} + \dots + \frac{1}{(2n-1)(2n+1)}
	      \]
	      where $n$ is any natural number.\\

	      The solution is that for any $n\in\NN$, we have
	      \[
		      \frac{1}{1\cdot 3} +\frac{1}{3\cdot 5} + \dots + \frac{1}{(2n-1)(2n+1)} = \frac{n}{2n+1}
	      \]
	      We will now prove this by induction.

	      \noindent\underline{Base case:} In the case $n=1$, we have
	      \[
		      \frac{1}{1\cdot 3} = \frac{1}{3}
	      \]
	      We therefore see that the claim holds for $n=1$.\\
	      \underline{Inductive Step:} Suppose that the claim is true for some $n\geq 1$. That is, suppose that equation
	      \[
		      \frac{1}{1\cdot 3} +\frac{1}{3\cdot 5} + \dots + \frac{1}{(2n-1)(2n+1)} = \frac{n}{2n+1}
	      \]
	      is true.

	      Then for $n+1$ we have
	      \begin{align*}
		        & \underbrace{\frac{1}{1\cdot 3} +\frac{1}{3\cdot 5} + \dots + \frac{1}{(2n-1)(2n+1)}}_{\text{induction hypothesis!}} + \frac{1}{(2(n+1)-1)(2(n+1)+1)} \\
		      = & \frac{n}{2n+1} + \frac{1}{(2(n+1)-1)(2(n+1)+1)}                                                                                                      \\
		      = & \frac{n+1}{2(n+1)+1}
	      \end{align*}
	      where the last equality follows from a series of algebraic manipulations.
	      \iffalse
		      \begin{align*}
			      \frac{n}{2n+1} + \frac{1}{(2(n+1)-1)(2(n+1)+1)}
			      = & \frac{n}{2n+1} + \frac{1}{(2n+1)(2(n+1)+1)} \\
			      = & \frac{n(2n+3)+1}{(2n+1)(2(n+1)+1)}          \\
			      = & \frac{2n^2+3n+1}{(2n+1)(2(n+1)+1)}          \\
			      = & \frac{2n^2+3n+1}{(2n+1)(2(n+1)+1)}          \\
			      = & \frac{(2n+1)(n+1)}{(2n+1)(2(n+1)+1)}        \\
			      = & \frac{n+1}{2(n+1)+1}                        \\
		      \end{align*}
	      \fi
	      The above shows that the statement holds for $n+1$. This concludes the induction step.

	\item Define $x_1 = 1, x_2=2$ and for every $n\geq 1$, $x_n = \frac{1}{2}(x_n+x_{n+1})$. Use the Principle of strong induction to show that $1\leq x_n \leq 2$ for all natural numbers $n$. \\

	      \noindent\underline{Base case:} We see that the statement is true for $n=1$ and $n=2$. Indeed, $1 \leq x_1\leq 2$ since $x_1=1$ and $1\leq x_2\leq 2$ since $x_2=2$.

	      \noindent\underline{Inductive step:} Suppose the statement holds true up to some $n\geq 2$. That is, $1\leq x_k \leq 2$ for all natural numbers $k \leq n$.

	      We want to show that $1\leq x_{n+1} \leq 2$. First, we show $x_{n+1}\geq 1$.

	      By the induction hypothesis, $x_{n-1} \geq 1$ and $x_n \geq 1$. It follows that
	      \[
		      x_{n-1} + x_n \geq 2
	      \]
	      Dividing either side by 2 yields
	      \[
		      \frac{x_{n-1} + x_n}{2}\geq 1
	      \]
	      But since $x_{n+1} = \frac{x_{n-1} + x_n}{2}$, the above shows that $x_{n+1}\geq 1$.

	      Now, we show that $x_{n+1} \leq 2$. By the induction hypothesis, $x_{n-1} \leq 2$ and $x_n \leq 2$. It follows that
	      \[
		      x_{n-1} + x_n \leq 2 + 2 = 4
	      \]
	      Dividing either side by 2 yields
	      \[
		      \frac{x_{n-1} + x_n}{2}\leq 2
	      \]
	      But since $x_{n+1} = \frac{x_{n-1} + x_n}{2}$, the above shows that $x_{n+1}\leq 2$.

	      We conclude that $1\leq x_{n+1} \leq 2$ as desired.

	\item Simplify the expression representing the following sets;
	      \begin{enumerate}
		      \item $A\setminus (B\setminus A)$,
		      \item $A\setminus (A\setminus B)$,
		      \item $A\cap (B\setminus A)$.
	      \end{enumerate}

	      The simplified version are;
	      \begin{enumerate}
		      \item $A\setminus (B\setminus A) = A$,
		      \item $A\setminus (A\setminus B) = A\cap B$,
		      \item $A\cap (B\setminus A) = \emptyset$.
	      \end{enumerate}

	\item Show that $A\subseteq B$ if and only if $A\cap B = A$. \\

	      Suppose first that $A\subseteq B$; we will show that $A\cap B = A$.

	      If $x\in A\cap B$, then $x\in A$ and $x\in B$. In particular, $x\in A$. This shows the inclusion $A\cap B\subseteq A$.

	      Conversely, let $x\in A$. Since we assumed that $A\subseteq B$, we also have $x\in B$. But then $x\in A$ and $x\in B$ which means that $x\in A\cap B$. This shows the converse inclusion $A\subseteq A\cap B$.

	      We conclude that $A\cap B = A$.

	      We have therefore showed that if $A\subseteq B$ then $A\cap B = A$.\\

	      We now show that if $A\cap B = A$ then $A\subseteq B$. So, we suppose that $A\cap B = A$. If $x\in A$ then $x\in A\cap B$. In particular, we have $x\in B$. We have thereby shown that if $x\in A$ then $x\in B$, that is $A\subseteq B$.

	      So we have assumed that $A\cap B = A$ and concluded that $A\subseteq B$.

	      This concludes the proof.

	\item Show that $A\cup B = A\cap C$ if and only if $B\subseteq A\subseteq C$.

	\item Define $A_n = \set{(n+1)k : k\in \NN}$ for each $n\in \NN$.
	      \begin{enumerate}
		      \item What is $A_1\cap A_2$?
		      \item Determine the sets $\bigcup_{n\in \NN}A_n$ and $\bigcap_{n\in \NN} A_n$.
	      \end{enumerate}

	      (a) First, note that the set $A_n$ is the set of all natural numbers divisible by $n+1$. So $A_1$ is all the even positive integers and $A_2$ is all the positive integer divisible by 3. So we claim that
	      \[
		      A_1\cap A_2 = \set{\text{positive integers divisble by 6}} = A_5
	      \]
	      To prove this equality, suppose first that $x\in A_1\cap A_2$, then there exists $k_1 \in \NN$ such that $x = 2k_1$ and $k_2\in \NN$ such that $x = 3k_2$. So,
	      \[
		      x=  2k_1 = 3k_2
	      \]
	      In particular, we claim that $k_2$ is even (if this were not the case, then $3k_2$ would be odd since the product of odd numbers is odd; but this is not possible). It follows that $k_2 = 3k$ for some $k\in \NN$. In particular, $x = 6k$. It follows from the definition of $A_5$ that $x\in A_5$. This proves the inclusion
	      \[
		      A_2\cap A_3\subseteq A_5
	      \]

	      Conversely, suppose that $x\in A_5$. Then $x = 6k$ for some $k\in \NN$. Writing $x = 2(3k)$ we see that $x\in A_1$ and writing $x = 3(2k)$ we see that $x\in A_2$. Thus, $x\in A_1\cap A_2$. This proves the inclusion $A_5\subseteq A_1\cap A_2$ whence we conclude that $A_2\cap A_3 = A_5$.


	      (b) We claim that
	      \[
		      \bigcup_{n\in \NN} A_n = \NN\setminus\set{1}
	      \]
	      First we show the inclusion
	      \[
		      \bigcup_{n\in \NN} A_n \subseteq \set{2,3, 4, \dots} = \NN\setminus\set{1}
	      \]
	      To this end, suppose $x\in\bigcup_{n\in \NN} A_n$. Then $x\in A_n$ for some $n\in \NN$. But then, $x = (n+1)k$ for some $k\in \NN$. Since $n\in \NN$, we have $n\geq 1$ and $n+1\geq 2$. It follows that $x\geq 2$. So, $x$ is a natural number and $x\neq 1$ whence $x\in \NN\setminus \set{1}$.

	      Conversely, consider any number $x\in \NN\setminus \set{1}$. Then since $x\geq 2$, it must be the case that $x-1 \in \NN$. But the, $x = [(x-1)+1]\cdot 1 \in A_{x-1}$. It follows that $x\in\bigcup_{n\in \NN} A_n$. This shows the inclusion
	      \[
		      \NN\setminus\set{1} \subseteq \bigcup_{n\in \NN} A_n
	      \]
	      and we conclude that equality must hold.

	      For the second part, we claim that
	      \[
		      \bigcap_{n\in \NN} A_n = \emptyset
	      \]
	      To see this, it suffices to suppose that there exists an element $x\in \bigcap_{n\in \NN} A_n$ and derive a contradiction. Indeed, by deriving a contradiction we show that it is impossible for $\bigcap_{n\in \NN} A_n$ to contain any number which means that $\bigcap_{n\in \NN} A_n = \emptyset$. Without further ado, suppose $x\in \bigcap_{n\in \NN} A_n$. Then $x\in A_n$ for all $n\in \NN$. In particular, it must be the case that $x \in A_{x}$. But then, we must have $x=  (x+1)k$ for some $k\in \NN$. In particular,
	      \[
		      x = (x+1)k \geq x+1 > x \implies x>x
	      \]
	      which is clearly a contradiction.

	\item Let us now put into practice the concept of the image and pre-image of a function. Define the function $f:\RR\setminus\set{0}\to \RR$ given by $f(x) = 1/x^2$. What is $f(E)$ where
	      \[
		      E = \set{x\in \RR : 1\leq x\leq 2} = [1,2]?
	      \]
	      What is $f^{-1}(G)$ where
	      \[
		      G = \set{x\in \RR :  1\leq x\leq 4} = [1, 4]?
	      \]

	      We have
	      \begin{align*}
		      f(E) & = \set{f(x) : x\in E}              \\
		           & =\set{1/x^2 : 1\leq x \leq 2}      \\
		           & =\set{y\in \RR : 1/4\leq y \leq 1} \\
		           & =[1/4, 1]
	      \end{align*}
	      Furthermore,
	      \begin{align*}
		      f^{-1}(G) & = \set{x \in \RR\setminus\set{0} :  f(x) \in G}                                      \\
		                & = \set{x \in \RR\setminus\set{0} :  1\leq f(x) \leq 4}                               \\
		                & = \set{x \in \RR\setminus\set{0} :  1\leq 1/x^2 \leq 4}                              \\
		                & = \set{x \in \RR\setminus\set{0} :  1\geq x^2 \geq 1/4}                              \\
		                & = \set{x \in \RR\setminus\set{0} :  1\geq x \geq 1/2 \text{ or } -1\leq x \leq -1/2} \\
		                & = [-1, -1/2] \cup [1/2, 1]
	      \end{align*}

	\item Let $f:A\to B$ and $g:B\to C$ be functions. Show that for every $X\subseteq C$ it holds that
	      \[
		      (g\circ f)^{-1}(X) = f^{-1}\left(g^{-1}(X)\right)
	      \]\\

	      To prove this, we begin by picking an arbitrary subset $X$ of $C$. Now, we have to show that
	      \[
		      (g\circ f)^{-1}(X) = f^{-1}\left(g^{-1}(X)\right)
	      \]
	      We begin by proving the first inclusion. To this end, consider any point $x\in (g\circ f)^{-1}(X)$. Then
	      \[
		      (g\circ f)(x) = g\left(f(x)\right) \in X
	      \]
	      But then, $f(x) \in g^{-1}(X)$. It follows that $x\in f^{-1}\left(g^{-1}(X)\right)$ as desired. So we have shown that any point in $(g\circ f)^{-1}(X)$ is also a point in $f^{-1}\left(g^{-1}(X)\right)$. That is,
	      \[
		      (g\circ f)^{-1}(X) \subseteq f^{-1}\left(g^{-1}(X)\right)
	      \]

	      To see that the reverse inclusion holds as well, consider any point $x\in f^{-1}\left(g^{-1}(X)\right)$. By definition, we have $f(x) \in g^{-1}(X)$. But then,
	      \[
		      g(f(x)) = (g\circ f )(x) \in X
	      \]
	      That is, $x\in (g\circ f)^{-1}(X)$. So we have that every point in $f^{-1}\left(g^{-1}(X)\right)$ is also in $x\in (g\circ f)^{-1}(X)$. Thus, the reverse inclusion
	      \[
		      f^{-1}\left(g^{-1}(X)\right) \subseteq (g\circ f)^{-1}(X)
	      \]
	      holds as well.

	      Since both inclusions hold, we have
	      \[
		      (g\circ f)^{-1}(X) = f^{-1}\left(g^{-1}(X)\right)
	      \]
	      Since the choice of $X\subseteq C$ was arbitrary, we see that the above equality holds for all subsets $X$ of $C$.

\end{enumerate}

\section*{Tutorial 3}

\begin{enumerate}[leftmargin=*]
	\item I wanted to clarify what I mean by the notation ``$:=$''. For instance, if I write $y:= f(x)$ then you can read it as ``define $y$ to be $f(x)$''. So, if you read a proof and you see, for instance, $a:=f\circ g(x)$, you do not have to wonder why the equality is valid. I am simply introducing a new variable and defining it that way.
	\item Consider the function
	      \[
		      f: \NN\to \RR,\quad x\mapsto \frac{1}{x}
	      \]
	      Is this function injective? surjective?

	      The function is injective but nor surjective.

	\item Is the function
	      \[
		      f: \RR\to\RR, \quad x\mapsto\sin(x)
	      \]
	      injective? surjective?

	      The function is not injective nor surjective.

	      How can we restrict the domain to make it injective?

	      Well, note that $f\mid_{[\pi/2, 3\pi/2]}$ which is the function
	      \[
		      f\mid_{[\pi/2, 3\pi/2]}: [\pi/2, 3\pi/2]\to\RR, \quad x\mapsto\sin(x)
	      \]
	      is injective. Furthermore, changing the image make $f\mid_{[\pi/2, 3\pi/2]}$ a bijection. That is, if we view $f\mid_{[\pi/2, 3\pi/2]}$ as a function from $[\pi/2, 3\pi/2]$ to $[-1,1]$ then it is bijective.

	\item Let $f:A\to B$ be an injective function and let $C, D \subseteq A$. Show that if $f(C)\subseteq f(D)$ then $C\subseteq D$.

	      Suppose that $f:A\to B$ is an injective function and the sets $C, D \subseteq A$ are such that $f(C)\subseteq f(D)$. We wish to show that $C\subseteq D$. To this end, suppose $x \in C$. By definition, we have $y:= f(x) \in f(C)$. Since $f(C)\subseteq f(D)$, it follows that $y\in f(D)$. This means that there exists $x^\prime \in D$ such that $f(x^\prime) = y$.

	      Note that we then have
	      \[
		      f(x^\prime) = y = f(x)
	      \]
	      Since $f$ is injective, this implies that $x = x^\prime$. In particular, $x\in D$ (since $x^\prime \in D$). We conclude that $C\subseteq D$.

	\item Give an example of a function $f:\RR\to \RR$ and subsets $C, D$ of $\RR$ such that $f(C) \subseteq f(D)$ but $C\not\subseteq D$.

	      Consider for instance the function $f(x)=x^2$ and the set $C = [1,2]$ and $D = [-2, -1]$.

	\item Show that for any function $f : A\to B$ and $C, D\subseteq A$ it holds that $f(C\cap D) \subseteq f(C)\cap f(D)$.

	      Fix an arbitrary element $y\in f(C\cap D)$. By definition, $y = f(x)$ for some $x\in C\cap D$. In particular, $x\in C$ so $y = f(x) \in f(C)$. Similarly, it follows from $x\in D$ that $y = f(x)\in f(D)$. Therefore, $y\in f(C)\cap f(D)$. Since $y$ was an arbitrary element of $f(C\cap D)$,we infer that $f(C\cap D)\subseteq f(C)\cap f(D)$.

	\item Show that if $f : A\to B$ is injective and $C, D\subseteq A$ then $f(C\cap D) = f(C)\cap f(D)$.

	      The inclusion $f(C\cap D)\subseteq f(C)\cap f(D)$ is known by the previous problem. Conversely, fix an arbitrary element $y\in f(C)\cap f(D)$. Since $y\in f(C)$, there exists $x_1\in C$ such that $y = f(x_1)$. Since $y\in f(D)$, there exists $x_2\in D$ such that $y = f(x_2)$. But then we have
	      \[
		      f(x_1) = y = f(x_2)
	      \]
	      The injectivity of $f$ implies that $x_1 = x_2$. In particular, $x_1 \in C$ and $x_1 = x_2 \in D$. Thus, $x_1 \in C\cap D$. So $y = f(x_1)$ is an element of $f(C\cap D)$. Since $y$ was an arbitrary element of $f(C)\cap f(D)$, this shows the inclusion $f(C)\cap f(D)\subseteq f(C\cap D)$ and completes the proof.

	\item Let $f:A\to B$ be a surjective function and $G, H \subseteq B$. Show that if $f^{-1}(G) \subseteq f^{-1}(H)$ then $G\subseteq H$.

	      Pick an arbitrary element $y\in G$. Since $f$ is surjective, there exists $x\in A$ such that $f(x) = y$. But then we have $x\in f^{1}(G) \subseteq f^{-1}(H)$. Since $x\in f^{-1}(H)$, it must hold that $y = f(x) \in H$. Since $y$ was an arbitrary element of $G$, this proves the inclusion $G\subseteq H$.

	\item Give an example of a function $f:\RR\to \RR$ and $G, H\subseteq \RR$ such that $f^{-1}(G)\subseteq f^{-1}(H)$ but $G\not\subseteq H$.

	      Consider $f(x) = x^2$, $G = (-\infty, 0]$ and $H = [0,1]$.

	\item Suppose $f:A\to B$ is a function. Show that if $f^{-1}(f(E)) = E$ for all $E\subseteq A$ then $f$ is injective.

	      We prove the statement by contrapositive. So, we suppose that $f$ is not injective and wish to show that there exists $E\subseteq A$ for which $f^{-1}(f(E)) \neq E$.

	      Since $f$ is injective, there exist $x_1, x_2 \in A$ with $x_1\neq x_2$ such that $f(x_1) = f(x_2) =: y$. Now, for $E = \set{x_1}$ we have
	      \[
		      f^{-1}(f(E)) = f^{-1}(\set{y}) \supseteq \set{x_1, x_2}
	      \]
	      In particular, we have $f^{-1}(f(E)) \neq E$. This concludes the proof.

	\item Suppose $f:A\to B$ is a function. Show that if $f\left(f^{-1}(H)\right) = H$ for all $H\subseteq B$ then $f$ is surjective.

	      As in the previous problem, we prove the statement by contrapositive. To this end, suppose that $f$ is not surjective. Then there exists $y\in B$ such that $f(x)\neq y$ for all $x\in A$. Define $H = \set{y}$ and note that
	      \[
		      f\left(f^{-1}(H)\right) = f(\emptyset) = \emptyset \neq H
	      \]
	      This proves the statement.

	\item Suppose $f:A\to B$ is a function. Show that if $f(A\setminus E) = f(A)\setminus f(E)$ for all $E\subseteq A$ then $f$ is injective.

	      We prove the statement by contrapositive. That is, we suppose that $f$ is not injective and wish to show that there exists $E\subseteq A$ for which $f(A\setminus E) \neq f(A)\setminus f(E)$.

	      Since $f$ is not injective, there exists $x_1, x_2\in A$ with $x_1\neq x_2$ such that $f(x_1) = f(x_2) =:y$. Now, consider $E = \set{x_1}$. Then
	      \[
		      f(A\setminus E) =  f(A)
	      \]
	      But $f(A) \setminus f(E) \subsetneq f(A)$. This concludes the proof.

	      \begin{rem}
		      As promised in the tutorial, I will take this opportunity to show why proving equality of sets while maintaining set notation is sometimes ``dangerous''.

		      The set $f(A)\setminus f(E)$ is, by definition, equal to
		      \[
			      \set{f(x) :x\in A}\setminus \set{f(x) : x\in E}
		      \]
		      Now, it might be tempting to claim that
		      \[
			      \set{f(x) :x\in A}\setminus \set{f(x) : x\in E} = \set{f(x) :x\in A,\, x\not\in E}
		      \]
		      But
		      \[
			      \set{f(x) :x\in A,\, x\not\in E} = f(A\setminus E)
		      \]
		      and as seen in the above exercise, the equality $f(A)\setminus f(E) = f(A\setminus E)$ does not necessarily hold.
	      \end{rem}

	\item Show that if $f:A\to B$ is surjective then there exists a function $g:B\to A$ such that
	      \[
		      f\circ g(y) = y \qquad \forall y\in B
	      \]

	      For each $y\in B$, there exists $x_y\in X$ such that $f(x_y) = y$. Now, define $g(y) = x_y$ for every $y\in B$. Then it is clear that $f\circ g(y) = y$ for every $y\in B$.

	      \begin{rem}
		      Note that $g$ is not necessarily the inverse function of $f$. For instance, consider the functions $f: \RR\to[0,\infty), \, f(x) = x^2$ and $g:[0, \infty) \to \RR,\, g(x) = \sqrt{x}$. Note that $f$ and $g$ satisfy the conditions of the exercise but $g$ is not the inverse function of $f$.
	      \end{rem}

	\item Suppose that $f:A\to B$ is a function. Show that if there exists $g:B\to A$ such that
	      \[
		      f\circ g(y) = y \qquad \forall y\in B
	      \]
	      then $f$ is surjective.

	      Fix an arbitrary element $y\in b$. Then $x:= g(y)$ satisfies $f(x) = y$. Thus, $f$ is surjective.

	\item Show that if $f:A\to B$ is injective then there exists a function $g:B\to A$ such that
	      \[
		      g\circ f(x) = x \qquad \forall x\in A
	      \]

	      Fix \underline{any} element $x_0\in A$ and define the function $g : B\to A$ by
	      \[
		      g(y) = \begin{cases}
			      x\in A \text{ such that } f(x) = y & \text{if }y\in f(A) \\
			      x_0                                & \text{otherwise}
		      \end{cases}
	      \]
	      Note that $g$ is as desired.

	      \begin{rem}
		      We note again that $g$ is not necessarily the inverse function of $f$. For instance, consider the functions $f: [0, \infty)\to \RR, \, f(x) = x^2$ and $g:\RR \to [0, \infty)$,
		      \[
			      g(x) = \begin{cases}
				      \sqrt{x} & x\geq 0 \\
				      1        & x<0
			      \end{cases}
		      \]
		      Note that $f$ and $g$ satisfy the conditions of the exercise but $g$ is not the inverse function of $f$.
	      \end{rem}

	\item Suppose that $f:A\to B$ is a function. Show that if there exists $g:B\to A$ such that
	      \[
		      g\circ f(x) = x \qquad \forall x\in A
	      \]
	      then $f$ is injective.

	      To see that $f$ is injective, suppose that $f(x_1) = f(x_2)$ for some $x_1, x_2\in A$. Then
	      \[
		      x_1 = g\circ f(x_1) = g(f(x_1)) = g(f(x_2)) = g\circ f(x_2) = x_2
	      \]
	      We conclude that $f$ is indeed injective.

	\item Suppose $A$ is an infinite set and $B\subseteq A$ a finite set. Show that $A\setminus B$ must also be infinite.

	      Suppose for a contradiction that $A\setminus B$ is finite. Then note that $A\setminus B$ and $B$ are disjoint finite sets whose union is $A$. By part (1) of exercise 3 in your third assignment, $A$ is finite. But this contradicts the assumption that $A$ is infinite.
\end{enumerate}

\section*{Tutorial 4}

\begin{enumerate}[leftmargin=*]
	\item We start with an exercise which isn't particularly difficult, but is a good example of how you have to use definitions. Remember that in order to prove something, you have to either use a theorem or directly use the definition. Sometimes, this makes things seem ``tautological''.

	      \emph{Prove that is non-empty set $T_1$ is finite if and only if there is a bijection from $T_1$ onto a finite set $T_2$.}
	      \begin{proof} In the proof, we always assume that $T_1$ is non-empty.

		      Suppose first that $T_1$ is finite. Then the identity function $\mathrm{Id}: T_1 \to T_1$ which is given by $x\mapsto x$ is clearly a bijection from $T_1$ onto the finite set $T_1$. This proves the first direction.

		      Conversely, suppose there exists a bijection $f:T_1\to T_2$ where $T_2$ is a finite set. Since $T_2$ is finite, the definition states that, for some $n\in\NN$ there exists a bijection $g:\NN_n \to T_2$. Now, let $f^{-1} : T_2 \to T_1$ denote the inverse function of $f$ (we can indeed find the inverse function since $f$ is bijective). Finally, the function $f^{-1}\circ g : \NN_n \to T_1$ is a bijection. By definition, this means that $T_1$ is a finite set (with $n$ elements).
	      \end{proof}

	\item \emph{Prove that is non-empty set $T_1$ is countably infinite if and only if there is a bijection from $T_1$ onto a countably infinite set $T_2$.}
	      \begin{proof} In the proof, we always assume that $T_1$ is non-empty.

		      Suppose first that $T_1$ is countably infinite. Then the identity function $\mathrm{Id}: T_1 \to T_1$ which is given by $x\mapsto x$ is clearly a bijection from $T_1$ onto the countably infinite set $T_1$. This proves the first direction.

		      Conversely, suppose there exists a bijection $f:T_1\to T_2$ where $T_2$ is a countably infinite set. Since $T_2$ is countably infinite, the definition states that there exists a bijection $g:\NN \to T_2$. Now, let $f^{-1} : T_2 \to T_1$ denote the inverse function of $f$ (we can indeed find the inverse function since $f$ is bijective). Finally, the function $f^{-1}\circ g : \NN \to T_1$ is a bijection. By definition, this means that $T_1$ is a countably infinite set.
	      \end{proof}

	\item Recall the following definitions;
	      \begin{def_}[Function]
		      Given sets $A, B$, a function from $A$ to $B$ is a set $f$ of ordered pairs in $A\times B$ such that for each $a\in A$, there exists a unique $b\in B$ with $(a,b)\in f$. The set $A$ is called the domain of $f$.
	      \end{def_}
	      \begin{def_}
		      If $f:A\to B$ is bijective (i.e. injective and surjective) then the function
		      \[
			      g:\set{(b,a) \in B\times A :(a,b) \in f}
		      \]
		      is called the inverse function of $f$.
	      \end{def_}
	      \emph{Show that if $f:A\to B$ is bijective, then the inverse function $g$ of $f$ is indeed a bijective function.}
	      \begin{proof}
		      First we show that $g$ is indeed a function. For every $b\in B$, since $f$ is bijective there exists $a\in A$ such that $(a,b) \in f$ whence $(b,a)\in g$. Since $f$ is injective, the choice of $a$ is unique. This shows that $g$ is indeed a function.

		      To see that $g$ is injective, suppose that $g(x_1) = g(x_2) =: a$. Then $(a, x_1), (a,x_2)\in f$ (i.e. $f(a) = x_1$ and $f(a) = x_2$). Since $f$ is a function, it follows that $x_1 = x_2$.

		      To see that $g$ is surjective, consider $a\in A$. Since $f$ is a function, there exists a unique $b\in B$ such that $(a,b)\in f$. In particular, $(b,a) \in g$. We conclude that $g$ is indeed surjective
	      \end{proof}

	\item Show that if $S$ and $T$ are countable (finite or infinite), then $S\cup T$ is also countable.

	      \begin{proof}
		      Suppose $S$ and $T$ are countable.

		      If both $S$ and $T$ are finite, then (by your assignment) $S\cup T$ is also finite. If $S$ is infinite but $T$ is finite, then there exist bijective functions $f:\NN\to S$ and $g:\NN_n \to T$. It follows that the function $h: \NN\to S\cup T$ given by
		      \[
			      h(x) = \begin{cases}
				      g(x)   & \text{if } x\leq n \\
				      f(x-n) & \text{if }x>n
			      \end{cases}
		      \]
		      is a surjection (not necessarily a bijection!). By theorem 1.3.10. in your textbook, $S\cup T$ is countable.
		      Similarly, if $S$ is finite and $T$ is infinite, we see that $S\cup T$ is countable.

		      Finally, suppose that $S$ and $T$ are both infinite. Then there exist bijective functions $f:\NN\to S$ and $g:\NN\to T$. It follows that the function $h:\NN\to S\cup T$ given by
		      \[
			      h(x) = \begin{cases}
				      g(x/2)     & \text{if $x$ is even} \\
				      f((x+1)/2) & \text{if $x$ is odd}
			      \end{cases}
		      \]
		      is a surjection (again, not necessarily a bijection). By theorem 1.3.10. in your textbook, $S\cup T$ is countable.
	      \end{proof}
	      \begin{proof}
		      We present an alternate proof in the case where $S$ and $T$ are both countably infinite. Let $f_1:\NN\to S$ and $f_2:\NN\to T$ be bijective functions. Now, consider the function $g: \NN\times \NN \to S\cup T$ given by
		      \[
			      g(n, m) = \begin{cases}
				      f_1(n) & \text{if } m=1   \\
				      f_2(n) & \text{if } m=2   \\
				      x_0    & \text{otherwise}
			      \end{cases}
		      \]
		      where $x_0$ is any point in $S\cup T$. Clearly, $g$ is a surjective function. Now, since $\NN\times\NN$ is countable, there exists a bijective function $h : \NN\to\NN\times \NN$. It follows that $g\circ h : \NN\to S\cup T$ is a surjective function. By theorem 1.3.10. in your textbook, $S\cup T$ is countable.
	      \end{proof}

	\item \emph{Prove that the collection $\mathcal{F}(\NN)$ of finite subsets of $\NN$ is countable.}

	      \begin{proof}
		      Suppose $S$ is a finite subset of $\NN$, then note that $S\subseteq \NN_m$ for some $m\in \NN$.

		      Now, for each $m\in\NN$ define $A_n$ to be the collection of all subsets of $\NN_{n}$, i.e. $A_n = \P(\NN_{n})$. A previous exercise in the Textbook (which we have not proven) states that $A_n$ is a finite set with $2^n$ elements.

		      Finally, writing
		      \[
			      \mathcal{F}(\NN) = \bigcup_{n\in\NN} A_n
		      \]
		      we see that $\mathcal{F}(\NN)$ is the countable union of countable sets and thus countable.
	      \end{proof}

	\item \emph{Prove that the function $f:\NN\to \ZZ$ given by
		      \[
			      f(k) = \begin{cases}
				      k/2      & \text{if $k$ is even} \\
				      -(k-1)/2 & \text{if $k$ is odd}
			      \end{cases}
		      \]
		      is a bijection.}

	      \begin{proof}
		      To see that $f$ is injective, suppose that $k_1, k_2\in \NN$ are such that $f(k_1) = f(k_2) =: y$. If $y \geq 1$ then we must have
		      \[
			      \frac{k_1}{2} = f(k_1) = y =f(k_2) = \frac{k_2}{2}
		      \]
		      It follows that $k_1= k_2$. In the case $y\leq 0$, we see again that $k_1 = k_2$.

		      To see that $f$ is surjective, suppose $n\in\ZZ$. If $n > 1$ then $n = f(2n)$. If $n \leq 0$, then $n = f(-2n+1)$.

		      We conclude that $f$ is bijective.
	      \end{proof}

	\item \emph{Construct a bijection from $\NN$ to the set $O$ of odd integers.}

	      \begin{proof}
		      The function $f:\NN \to O$ given by
		      \[
			      f(k) = \begin{cases}
				      -(k-1) & \text{if $k$ is even} \\
				      k      & \text{if $k$ is odd}
			      \end{cases}
		      \]
		      is a bijection.
	      \end{proof}

	\item If $a,b\in\RR$, use the Algebraic properties of $\RR$ to show that if $a+b = 0$ then $b = -a$.

	      \begin{proof}
		      Suppose $a + b = 0$. Then since and $a + (-a) = 0$ we may write
		      \begin{align*}
			      b
			      = & b + 0 \tag{A3 : existence of a zero element} \\
			      = & b + (a+(-a))                                 \\
			      = & (b+a) + (-a) \tag{A2 : associativity}        \\
			      = & (a+b) + (-a) \tag{A1 : commutativity}        \\
			      = & 0 + (-a)                                     \\
			      = & -a \tag{A3 : existence of a zero element}
		      \end{align*}
		      So we have $b = -a$ and conclude that additive inverses are unique.
	      \end{proof}

	\item Using the algebraic properties of $\RR$, show that $(-1)\cdot a = -a$.

	      \begin{proof}
		      We write;
		      \begin{align*}
			      (-1)\cdot a
			      = & (-1)\cdot a + 0\tag{A3 : existence of a zero element}     \\
			      = & (-1)\cdot a + (a + (-a))                                  \\
			      = & ((-1)\cdot a + a) + (-a)\tag{A2 : associativity}          \\
			      = & ((-1)\cdot a + 1\cdot a) + (-a) \tag{M3: existence 1}     \\
			      = & ((-1) + 1)\cdot a + (-a) \tag{D :  distributive property} \\
			      = & 0\cdot a + (-a)                                           \\
			      = & -a \tag{Theorem 2.1.2 (c)}
		      \end{align*}
	      \end{proof}

	\item Show that
	      \[
		      \left[\frac{1}{2}(a+b)\right]^2 \leq \frac{1}{2}\left(a^2+b^2\right)
	      \]
	      \begin{proof}
		      First notice that for all $a,b \in \RR$ we have
		      \[
			      a^2-2ab+b^2 = (a-b)^2  \geq 0
			      \implies 2ab \leq a^2+b^2
		      \]
		      It follows that
		      \begin{align*}
			      \left[\frac{1}{2}(a+b)\right]^2
			       & =\frac{1}{4}\left(a^2+\underbrace{2ab}_{\leq a^2+b^2}+b^2\right) \\
			       & \leq \frac{1}{4}\left(a^2 + a^2+b^2 + b^2\right)                 \\
			       & =\frac{1}{2}\left(a^2+b^2\right)
		      \end{align*}
	      \end{proof}

	\item Show that if $a_1, \dots a_n$ are real positive numbers such that $a_1\cdots a_n = 1$ then
	      \[
		      1 \leq \frac{a_1 + a_2 + \dots + a_n}{n}
	      \]

	      \begin{proof} Equivalently, we will prove that $a_1+a_2+\dots +a_n \geq n$.
		      We prove this by induction. The case $n=1$ is clear. Now, suppose the statement holds for some $n\in \NN$ and let $a_1, \dots, a_{n+1}$ be real positive numbers. If $a_1 = a_2 = \dots = a_{n+1} = 1$, then the inequality is clear. Otherwise, we may suppose without loss of generality that $a_{n+1} > 1$ and $a_n < 1$. Now, define $y = a_na_{n+1}$. By the induction hypothesis,
		      \[
			      a_1+a_2+\dots +a_{n-1} + y \geq n
		      \]
		      On the other hand, notice that $(1-a_n)(a_{n+1} - 1)> 0$ implies that $a_{n+1} + a_n \geq y + 1$. Using this and the above inequality we see that
		      \begin{align*}
			      a_1+a_2+\dots +a_{n-1} + a_n + a_{n+1}
			       & \geq a_1 + a_2 + \dots + a_{n-1} + y + 1 \\
			       & \geq n+1
		      \end{align*}
		      This completes the proof.
	      \end{proof}

	\item Prove the general Arithmetic-Geometric Mean Inequality. That is, show that for all real positive numbers $a_1, \dots, a_n$ there holds
	      \[
		      \left(a_1a_2\cdots a_n\right)^{1/n} \leq \frac{a_1+ a_2 + \dots + a_n}{n}
	      \]
	      \begin{proof} We consider two possible cases

		      \underline{Case 1}: Suppose first that $a_1 a_2\cdots a_n = 1$. Then by the previous problem we have
		      \[
			      \left(a_1 a_2\cdots a_n\right)^{1/n} = 1 \leq \frac{a_1 + \dots  + a_n}{n}
		      \]
		      \underline{Case 1}: If $a_1 a_2\cdots a_n \neq 1$ then define $A = a_1 a_2\cdots a_n$. For each $1\leq k\leq n$ we also define
		      \[
			      x_k = \frac{a_k}{A^{1/n}}
		      \]
		      Now, notice that
		      \[
			      x_1x_2\cdots x_n = \frac{a_1a_1 \cdots a_n}{A} = 1
		      \]
		      In particular, $x_1, x_2, \dots, x_n$ are as in Case 1 and we have
		      \[
			      \left(x_1x_1\cdots x_n\right)^{1/n} \leq \frac{x_1 + x_2 + \cdots +x_n}{n}
		      \]
		      But the above inequality is precisely
		      \[
			      1 \leq \frac{a_1 + \dots + a_n}{A^{1/n} n}
		      \]
		      Recalling that $A = a_1 a_2\cdots a_n$, we see that
		      \[
			      \left( a_1 a_2\cdots a_n\right)^{1/n}\leq \frac{a_1 + \dots + a_n}{n}
		      \]
		      as was desired.
	      \end{proof}

\end{enumerate}

\section*{Tutorial 5}

\begin{enumerate}[leftmargin=*]
	\item Prove that $\NN\times \NN$ is countable.

	      \begin{proof}
		      It suffices to show that there exists an injective function from $\NN\times \NN$ to $\NN$. For this, we define the function $f:\NN\times\NN\to\NN$ by
		      \[
			      f(n,m) = 2^n3^m
		      \]
		      and show that it is injective.

		      Suppose that $f(n_1, m_1) = f(n_2, m_2)$, then we have
		      \[
			      2^{n_1}3^{m_1} = 2^{n_2}3^{m_2}
		      \]
		      We know consider the 4 possible cases;
		      \begin{enumerate}
			      \item If $n_1 \geq n_2$ and $m_1 \geq m_2$ then we obtain
			            \[
				            2^{n_1-n_2}3^{m_1 - m_2} = 1
			            \]
			            Since both are integers, we must have $2^{n_1-n_2} = 1$ and $3^{m_1-m_2} = 1$. It follows that $n_1=n_2$ and $m_1=m_2$.
			      \item In the same way as above,we can see that if  $n_1 \leq n_2$ and $m_1 \leq m_2$ then $n_1=n_2$ and $m_1=m_2$.
			      \item If $n_1\geq n_2$ and $m_1 \leq m_2$ then we have
			            \[
				            2^{n_1 - n_2} = 3^{m_2-m_1}
			            \]
			            But then, it must be the case that $n_1 - n_2 = m_2-m_1 = 0$. That is, $n_1=n_2$ and $m_1=m_2$.
			      \item The case $n_1\leq n_2$ and $m_1 \geq m_2$ is handled similarly.
		      \end{enumerate}
	      \end{proof}

	\item Prove by induction that
	      \[
		      \sum_{k=0}^n \binom{n}{k} = 2^n
	      \]

	      \begin{proof}
		      Base case is clear. For the induction step write
		      \begin{align*}
			      \sum_{k=0}^{n+1} \binom{n+1}{k}
			       & =\binom{n+1}{0}+\binom{n+1}{n+1} + \sum_{k=1}^n\binom{n+1}{k} \\
			       & =1+1+\sum_{k=1}^n\left[\binom{n}{k} + \binom{n}{k-1}\right]   \\
			       & =\sum_{k=0}^n\binom{n}{k} + 1+\sum_{k=0}^{n-1} \binom{n}{k}   \\
			       & =2\cdot 2^n = 2^{n+1}
		      \end{align*}
	      \end{proof}

	\item Let $f:A\to B$, $E\subseteq A$ and $H\subseteq B$. Show that $E\subseteq f^{-1}f(E)$ and $f(f^{-1}(H))\subseteq H$.

	\item Show that if $f:A\to B$ is injective and $E\subseteq A$ then $E= f^{-1}f(E)$.

	\item Show that if $f:A\to B$ is surjective and $H\subseteq B$ then $f(f^{-1}(H)) = H$.

	\item Show that the countable union of countable sets is countable.

	\item Show that the following are equivalent;
	      \begin{enumerate}
		      \item $A$ is countable,
		      \item There exists a surjection $\NN\to A$,
		      \item There exists an injection $A\to \NN$.
	      \end{enumerate}

\end{enumerate}

\section*{Tutorial 6}

\begin{enumerate}[leftmargin=*]
	\item Fix $a\in \RR$ and $\epsilon > 0$. Show that for any $x,y\in V_\epsilon(a)$ there holds
	      \[
		      \abs{x-y} < 2\epsilon
	      \]

	      \begin{proof}
		      This is a consequence of the triangle inequality. Recall that
		      \[
			      V_\epsilon(a) = \set{z\in \RR : \abs{z-a} < \epsilon}
		      \]
		      Thus, if $x, y\in V_\epsilon(a)$ then we have
		      \[
			      \abs{x-y}\leq \abs{x-a} + \abs{a-y} < \epsilon + \epsilon = 2\epsilon
		      \]
	      \end{proof}

	\item Suppose $a\in \RR$ is such that $a<\epsilon$ for all $\epsilon>0$. Show that $a \leq 0$.

	      \begin{proof}
		      Suppose for a contradiction that $a > 0$. Then picking $\epsilon = a > 0$ we see that
		      \[
			      a < \epsilon = a
		      \]
		      which is absurd.
	      \end{proof}

	\item\label{p4} Show that for all $\epsilon > 0$ there exists $k\in \NN$ such that $\frac{1}{2^k} < \epsilon$.

	      \begin{proof}
		      First notice that for all $k\in \NN$ there holds
		      \[
			      k < 2^k
		      \]
		      One can prove this by induction or using Bernoulli's inequality.

		      Now, fix $\epsilon>0$. By the Archimedean property, there exists $k\in\NN$ such that $k >1/\epsilon$. Then
		      \[
			      \frac{1}{2^k} <\frac{1}{k} <\epsilon.
		      \]
		      Since $\epsilon$ was arbitrary, this concludes the proof.
	      \end{proof}

	\item Show that the set $S = \set{\frac{n}{2^k} : k,n\in\NN}$ is dense in $[0,\infty)$. That is, show that for all $x,y\in [0,\infty)$ with $x<y$ there exists $s\in S$ such that
	      \[
		      x<s<y
	      \]

	      \begin{proof}
		      Let $x<y$ be arbitrary points in $[0,\infty)$.

		      By the previous problem, there exists $k\in \NN$ such that
		      \[
			      \frac{1}{2^k} < y-x
		      \]
		      Consider now the set
		      \[
			      A = \set{n \in \NN : \frac{n}{2^k} > x}
		      \]
		      By the Archimedean property, there exists $N\in \NN$ such that $N > 2^kx$. It follows that $A$ is non-empty. Thus, the well ordering property of $\NN$ implies that $A$ has a least element $n_0$. We claim that
		      \[
			      x < \frac{n_0}{2^k} < y
		      \]
		      The first inequality follows from the fact that $n_0 \in A$. If $n_0 = 1$ then we are done since
		      \[
			      \frac{n_0}{2^k} = \frac{1}{2^k} < y-x < y
		      \]
		      Otherwise, $n_0 \geq 2$ is the least element of $A$. It follows that $n_0 - 1\not\in A$. That is,
		      \[
			      \frac{n_0 - 1}{2^k} \leq x
		      \]
		      Ergo,
		      \[
			      y = x + (y-x) > \frac{n_0 - 1}{2^k}  + \frac{1}{2^k} = \frac{n_0}{2^k}
		      \]
	      \end{proof}

	\item Let $A, B\subseteq \RR$ be two non-empty sets. Show that if $A\cap B$ is non-empty then
	      \[
		      \inf(A\cap B) \geq \max\set{\inf(A), \inf(B)}
	      \]
	      \begin{proof}
		      It suffices to notice that both $\inf(A)$ and $\inf(B)$ are lower-bounds for $A\cap B$. Therefore, $\inf(A\cap B) \geq \inf(A)$ and $\inf(A\cap B) \geq \inf(B)$. But this implies that
		      \[
			      \inf(A) \geq \max\set{\inf(A), \inf(B)}
		      \]
	      \end{proof}

	\item Suppose $A\subseteq \RR$ is a non-empty set. Show that if $A$ has a maximum $M$, then it also has a supremum and $\sup A = M$. Similarly, if $A$ has a minimum $m$ then it also has an infimum and $\inf A = m$.

	      \begin{proof}
		      We will only prove the first part. Suppose $A\subseteq \RR$ has a maximal element $M$. Then by definition, $M\in A$ and $a\leq M$ for all $a\in A$. In particular, the set $A$ is bounded above whence $A$ has a supremum.

		      It remains to show that $\sup A = M$. If $u$ is an upper bound for $A$ then
		      \[
			      a\leq u\qquad\forall a\in A
		      \]
		      In particular, $M\leq u$ since $M\in A$.

		      We conclude that $M$ is an upper bound which is less than or equal to any other upper bound for $A$. That is, $M$ is the supremum of $A$.
	      \end{proof}



	\item What are the infimum and supremum (if they exist) of the set
	      \[
		      S = \set{\frac{1}{2^k} : k\in \NN}.
	      \]
	      For this problem, you may use the following result; if $A\subseteq \RR$ is a non-empty set then the following are equivalent
	      \begin{enumerate}
		      \item $u$ is the infimum of $A$,
		      \item $u$ is a lower bound for $A$ and for all $\epsilon>0$ there exists $a\in A$ such that $a-\epsilon < u$.
	      \end{enumerate}

	      \begin{proof}
		      First, notice that the infimum and supremum exist since $S$ is bounded below by 0 and bounded above by 1. First, we claim that
		      \[
			      \sup S = \frac{1}{2}
		      \]
		      By the previous problem, it suffices to show that $\frac{1}{2}$ is the maximum of $S$. First, it is clear that $\frac{1}{2} \in S$. Moreover, if $s\in S$ then there exists $k\in\NN$ such that $s = \frac{1}{2^k}$. Since $k \geq 1$, we have
		      \[
			      s = \frac{1}{2^k} \leq \frac{1}{2}
		      \]
		      We conclude that $\sup S = \frac{1}{2}$.

		      Now, we claim the the infimum of $S$ is 0. Clearly, 0 is a lower bound for $S$. Furthermore, by problem \ref{p4}, for all $\epsilon > 0$, there exist $k\in \NN$ such that
		      \[
			      \frac{1}{2^k} <\epsilon
		      \]
		      Since $\frac{1}{2^k}\in S$, it follows that $\sup S = 0$.
	      \end{proof}

	\item What are the infimum and supremum (if they exist) of the set
	      \[
		      S = \set{\frac{1}{n} - \frac{1}{m} : n,m\in \NN}.
	      \]

	      \begin{proof}
		      First note that all elements in $S$ are bounded above and below. Indeed, given $n,m\in \NN$ we have
		      \[
			      -1 \leq \frac{1}{n} - \frac{1}{m} \leq 1
		      \]
		      We will prove that $\sup S = 1$. A similar argument shows that $\inf S = -1$.

		      Since we already now that 1 is an upper bound, it suffices to show that for all $\epsilon > 0$, there exists $s\in S$ such that
		      \[
			      s > 1-\epsilon
		      \]
		      To this end, fix $\epsilon > 0$. By the Archimedean property, there exists $m\in\NN$ such that
		      \[
			      \frac{1}{m} < \epsilon
		      \]
		      Then
		      \[
			      s:= 1 - \frac{1}{m} > 1-\epsilon
		      \]
		      Since $s\in S$, this concludes the proof.
	      \end{proof}

	\item Let $S$ be a non-empty subset of $\RR$. Show that $u$ is an upper bound for $S$ if and only if for all $t>u$, $t\not\in S$.

	      \begin{proof}
		      Suppose first that $u$ is an upper bound for $S$. The all $s\in S$ satisfy $s\leq u$. In particular, if $t>u$ then $t\not\in S$. Conversely, suppose that for all $t>u$, $t\not\in S$. Then for all $s\in S$, there holds that $s$ is not greater than $u$. That is, $s\leq u$. This shows that $u$ is an upper bound.

		      Note that the statement ``if $t>u$ then $t\not\in S$'' is the contra-positive of (and thus logically equivalent to) the statement ``If $s\in S$ then $s\leq u$''. Since the latter is the definition of an upper bound, we see that  $u$ is an upper bound for $S$ if and only if for all $t>u$, $t\not\in S$.
	      \end{proof}

\end{enumerate}

\section*{Tutorial 7}

Before beginning the exercises, we recall the following facts about open/closed sets;
\begin{enumerate}[label=(\alph*)]
	\item An arbitrary union of open sets is open,
	\item An arbitrary intersection of closed sets is closed,
	\item A finite intersection of open sets if open,
	\item A finite union of closed sets is closed.
\end{enumerate}

\begin{enumerate}[leftmargin=*]
	\item Are the following sets open or closed? Explain.
	      \begin{enumerate}[label=(\alph*)]
		      \item The set of rational numbers $\QQ$.
		      \item The set of irrational numbers $\QQ^\prime$.
		      \item The natural numbers $\NN$.
		      \item A set of the form $\set{x}$ for some $x\in \RR$.
		      \item A finite subset of $\RR$.
		      \item The set $(0,1) \cup [2, 3]$.
	      \end{enumerate}
	      \begin{proof}[Solution]$\,$

		      \hspace{0.5cm} (a) The rationals are neither open nor closed. First, to see that the rationals are not open, we suppose for a contradiction that $\QQ$ is open. Since $0\in\QQ$, there exists $\epsilon > 0$ such that $V_\epsilon(0)\subseteq \QQ$. Now, we show two possible ways to obtain a contradiction.

		      First, there exists a bijection
		      \[
			      f: [0,1] \to V_\epsilon(0) = (-\epsilon, \epsilon), \qquad
			      x\mapsto 2x - \epsilon
		      \]
		      Since $[0,1]$ is uncountable, it follows that $V_\epsilon(0)$ is also uncountable. On the other hand, since $\QQ$ is countable and $V_\epsilon(0) \subseteq \QQ$, the set $V_\epsilon(0)$ must be countable. This is a contradiction.

		      The second way to obtain a contradiction is to notice that (by density of the irrationals) there must exist an irrational number in $V_\epsilon(0)$. But this contradicts the assumption that $V_\epsilon(0)\subseteq \QQ$.

		      Furthermore, the rationals are not closed. To show this, we must prove that the irrationals $\QQ^\prime$ are not open. Suppose for a contradiction that $\QQ^\prime$ is open and fix $x\in \QQ^\prime$. Since $\QQ^\prime$ is open, there exists $\epsilon > 0$ such that $V_\epsilon(x) \subseteq \QQ^\prime$. But since the rationals are dense, there exists a rational number in $V_\epsilon(x)$. This is a contradiction.


		      \hspace{0.5cm}  (b) The irrationals are neither closed nor open. This follows from the fact that $\QQ$ is neither open nor closed.

		      \hspace{0.5cm}  (c) The natural numbers are closed. Indeed, we have
		      \[
			      \RR\setminus\NN = (-\infty, 1) \bigcup_{n=1}^\infty (n, n+1)
		      \]
		      which is open since the union of open sets is open.

		      \hspace{0.5cm}  (d) For any $x\in \RR$, we have that
		      \[
			      \RR\setminus\set{x} = (-\infty, x)\cup (x, \infty)
		      \]
		      is open. It follows that $\set{x}$ is closed.

		      \hspace{0.5cm} (e) Any finite set is closed. Indeed, a finite set is a finite union of singletons. As shown in (d), singletons are closed. It follows that finite sets are the finite union of closed sets and therefore closed.

		      \hspace{0.5cm}  (f) The set $(0,1)\cup [2,3]$ is neither open nor closed. Here, we only show that it is not open.

		      Suppose for a contradiction that $(0,1)\cup [2,3]$ is open. Then there exists $\epsilon>0$ such that $V_\epsilon(3) \subseteq (0,1)\cup [2,3]$. But $V_\epsilon(3)$ contains the point $3+\epsilon/2$ which is not in $(0,1)\cup [2,3]$.

	      \end{proof}


	\item Show that
	      \[
		      A:= \bigcap_{n=1}^\infty \left(-\frac{1}{n}, \frac{1}{n}\right) = \set{0}
	      \]

	      \begin{proof}
		      To see this, we first show that $\set{0}\subseteq A$. Indeed, $0\in (-\frac{1}{n}, \frac{1}{n})$ for all $n\in \NN$ whence $0\in A$.

		      Conversely, suppose $a$ is any element of $A$. If $a>0$ then, by the Archimedean property, there exists $k\in \NN$ with $k \geq 1/a$. It follows that $a \geq 1/k$ whence
		      \[
			      a\not\in \left(-\frac{1}{k}, \frac{1}{k}\right)
		      \]
		      But this contradicts the assumption that $a\in A$. It must therefore hold that $a \leq  0$.

		      Similarly, if $a<0$ then we may find $k\in \NN$ satisfying $k \geq -1/a$. But then $a \leq -1/k$ whence
		      \[
			      a\not\in \left(-\frac{1}{k}, \frac{1}{k}\right)
		      \]
		      which contradicts the assumption that $a\in A$. We must therefore have $a\geq 0$.

		      So, if $a\in A$ then we see that $a\geq 0$ and $a\leq 0$. That is, $a=0$. This shows that $A\subseteq \set{0}$.
	      \end{proof}

	\item Show that
	      \[
		      A:=\bigcup_{n=1}^\infty\left[-1+\frac{1}{n}, 1-\frac{1}{n}\right] = (-1,1)
	      \]
	      \begin{proof}
		      Suppose first that $a\in A$. Then there exists $k\in \NN$ such that
		      \[
			      a\in \left[-1+\frac{1}{k}, 1-\frac{1}{k}\right]
			      \implies
			      -1+\frac{1}{k} \leq a \leq 1-\frac{1}{k}
			      \implies -1 < a < 1
		      \]
		      That is, we have $a\in (-1,1)$. This shows that inclusion $A\subseteq (-1,1)$.

		      Conversely, suppose that $a\in (-1,1)$. Define
		      \[
			      \epsilon = \min\set{1-a, 1+a}
		      \]
		      That is, $\epsilon$ is such that
		      \[
			      -1 + \epsilon \leq a \leq 1-\epsilon
		      \]
		      By the Archimedean property, there exists $k\in\NN$ such that $k \geq 1/\epsilon$. In particular, $1/k \leq \epsilon$. Ergo,
		      \[
			      -1 + \frac{1}{k} \leq a \leq 1-\frac{1}{k} \implies a\in \left[-1+\frac{1}{k}, 1-\frac{1}{k}\right]
		      \]
		      We therefore have $a\in A$. This proves the inclusion $(-1,1)\subseteq A$.
	      \end{proof}

	\item Show that
	      \[
		      A:= \bigcup_{n=1}^\infty\left[1-\frac{1}{n}, n\right] = [0, \infty)
	      \]

	      \begin{proof}
		      Suppose first that $a\in A$. Then $a\in [1-1/n, n]$ for some $n\in \NN$. In particular,
		      \[
			      0 \leq 1-\frac{1}{n} \leq a
		      \]
		      That is, $a\in [0, \infty)$. Conversely, suppose that $a\in [0, \infty)$. This means that $a\geq 0$. If $0 \leq a \leq 1$ then we have
		      \[
			      a \in [0, 1] = \left[1-\frac{1}{1}, 1\right] \subseteq A
		      \]
		      On the other hand, if $a > 1$ then (by the Archimedean property) there exists $n \in \NN$ such that $n \geq a$. Therefore,
		      \[
			      1 - \frac{1}{n} < 1 < a \leq n \implies a \in \left[1-\frac{1}{n}, n\right] \subseteq A
		      \]
		      In either case, we deduce that $a\in A$.
	      \end{proof}
	\item Show that for any $b\in \RR$,
	      \[
		      \lim_{n\to\infty}\frac{b}{n} = 0
	      \]

	      \begin{proof}
		      Given $\epsilon > 0$, we know (by the Archimedean Property) that there exists $K\in \NN$ such that
		      \[
			      K > \frac{\abs{b}}{\epsilon}
		      \]
		      Then for all $n\geq K$ there holds
		      \[
			      \abs{\frac{b}{n} - 0} = \frac{\abs{b}}{n} \leq \frac{\abs{b}}{K} < \epsilon
		      \]
		      Ergo, $\lim_{n\to\infty}\frac{b}{n} = 0$.
	      \end{proof}

	\item Show that $x_n \to 1$ where
	      \[
		      x_1 = 0,
		      \quad\text{and}\quad
		      x_n = \frac{n^2}{n^2 - 1} \,\,\,\,\,\forall n \geq 2
	      \]

	      \begin{proof}
		      First, notice that for all $n\in\NN, n\geq 2$ there holds
		      \[
			      \abs{\frac{n^2}{n^2-1} - 1} = \abs{\frac{n^2 - n^2 + 1}{n^2 - 1}} = \frac{1}{n^2 - 1}
		      \]
		      Now, given $\epsilon>0$ we may find $K\in \NN$ such that
		      \[
			      K > \frac{1}{\epsilon}
		      \]
		      Without loss of generality, we may suppose that $K\geq 2$. Then for all $n\geq K$,
		      \[
			      \abs{x_n -1} =  \frac{1}{n^2 - 1} = \frac{1}{(n-1)(n+1)} \leq \frac{1}{n+1} < \frac{1}{K} < \epsilon
		      \]
	      \end{proof}

	\item Show that if $x_n \to x$ then $\abs{x_n} \to \abs{x}$.

	      \begin{proof}
		      Indeed, for $\epsilon > 0$, there exists $K\in \NN$ such that
		      \[
			      \abs{x_n - x} < \epsilon \qquad\forall n\geq K
		      \]
		      In particular,
		      \[
			      \abs{\abs{x_n} - \abs{x}} \leq \abs{x_n - x}< \epsilon \qquad\forall n\geq K
		      \]
	      \end{proof}

	      \begin{rem}
		      Note that the converse is not always true. Indeed, consider the sequence $x_n = (-1)^n$. Then $x_n$ does not converge while $\abs{x_n} \to 1$.
	      \end{rem}

	\item Show that if $(x_n)$ is a sequence such that $\abs{x_n} \to 0$ then $x_n \to 0$

	      \begin{proof}
		      Indeed, for all $\epsilon > 0$, there exists $K\in \NN$ such that
		      \[
			      \abs{x_n - 0} = \abs{\abs{x_n} - 0} < \epsilon
		      \]
		      for all $n\geq K$.
	      \end{proof}
\end{enumerate}

\section*{Tutorial 8}

\begin{enumerate}[leftmargin=*]
	\item Show that
	      \[
		      \lim{ \frac{2n}{n+1}} = 2.
	      \]
	      \begin{proof}
		      Let $\epsilon > 0$ be given, we must find $N \in \NN$ such that\[
			      \abs{ \frac{2n}{n+1} - 2 } < \epsilon, \quad \forall n \geq N.
		      \]
		      First, we calculate for all $n \in \NN$:
		      \begin{align*}
			      \abs{ \frac{2n}{n+1} - 2 } = \abs{ \frac{2n - 2(n+1)}{n+1} } = \frac{2}{n+1} \leq \frac{2}{n}.
		      \end{align*}
		      Here, one could note that $1/n \to 0$ (as demonstrated in the previous tutorial). It then follows from Theorem 1.3.10. in the Textbook that the limit is 2.

		      Alternatively, by the Archimedean property, we can find $N \in \NN$ such that $N > \frac{2}{\epsilon}$. Then, for all $n \geq N$ there holds
		      \[
			      \frac{2}{n} \leq \frac{2}{N} < 2 \cdot \frac{\epsilon}{2} = \epsilon.
		      \]
		      Hence, if $n \geq N$, we see that
		      \[
			      \abs{ \frac{2n}{n+1} - 2 } \leq  \frac{2}{n} < \epsilon,
		      \]
		      which is what had to be shown.
	      \end{proof}
	\item Prove that\[
		      \lim\frac{3n+1}{2n+5} = \frac{3}{2}.
	      \]
	      \begin{proof}
		      Let $\epsilon > 0$ be given, we seek $N \in \NN$ such that\[
			      \abs{ \frac{3n+1}{2n+5} - \frac{3}{2} } < \epsilon
		      \]
		      for all $n \geq N$. Naturally, for each $n \in \NN$, we have the following estimate:
		      \begin{align*}
			      \abs{ \frac{3n+1}{2n+5} - \frac{3}{2} } = \abs{ \frac{2(3n+1) - 3(2n+5)}{2(2n+5)}}
			       & = \abs{ \frac{2-15}{2(2n+5)} } \\
			       & = \frac{13}{2(2n+5)}           \\
			       & \leq \frac{4}{n}.
		      \end{align*}
		      Here, we can again apply Theorem 1.3.10. to conclude the result. Otherwise, choose $N \in \NN$ so large that $\frac{1}{N} < \frac{\epsilon}{4}$, if $n \geq N$ the above implies that
		      \begin{align*}
			      \abs{ \frac{3n+1}{2n+5} - \frac{3}{2} }  \leq \frac{4}{n} \leq \frac{4}{N} < 4 \cdot \frac{\epsilon}{4} = \epsilon.
		      \end{align*}
		      By definition, this means that
		      \[
			      \frac{3n+1}{2n+5}\xrightarrow{ n \to \infty} \frac{3}{2}.
		      \]
	      \end{proof}
	\item Show that\[
		      \lim{ \frac{n^2-1}{2n^2 + 3}}=  \frac{1}{2}.
	      \]
	      \begin{proof}
		      Following the outline in the remark, let $\epsilon > 0$ and consider the expression:
		      \begin{align*}
			      \abs{ \frac{n^2-1}{2n^2 + 3} - \frac{1}{2}} & = \abs{ \frac{2(n^2 - 1) - (2n^2+3)}{2(2n^2+3)} } \\
			                                                  & = \abs{ \frac{5}{2(2n^2 + 3)} }                   \\
			                                                  & \leq \frac{4}{n^2}.
		      \end{align*}
		      Now, it will be easier to find our candidate for $N$. Let $N \in \NN$ be such that $$N > \frac{2}{\sqrt{\epsilon}},$$ which we know to exist by the Archimedean property. If $n \geq N$, then our estimate above tells us that
		      \begin{align*}
			      \abs{ \frac{n^2-1}{2n^2 + 3} - \frac{1}{2}}  \leq \frac{4}{n^2} \leq \frac{4}{N^2} < 4 \cdot \frac{\epsilon}{4} = \epsilon.
		      \end{align*}
		      This is what had to be shown.
	      \end{proof}
	\item Let us prove that
	      \begin{equation}\label{eq:lim4-1}
		      \lim{\frac{2^n}{n!}} = 0.
	      \end{equation}
	      \begin{proof}
		      To do this, we will show that
		      \[
			      0 < \frac{2^n}{n!} \leq 2\left(\frac{2}{3}\right)^{n-2}, \quad \forall n \geq 2.
		      \]
		      If this indeed true, then the squeeze theorem would imply that
		      \[
			      \lim{\frac{2^n}{n!}} = 0.
		      \]
		      Thus, we are reduced to verifying the aforementioned inequality. For this, we shall argue by induction. As a base case, let us check the inequality for $n=2$:
		      \[
			      0 < \frac{2^2}{2!} = 2 \quad \text{and} \quad 2\left(\frac{2}{3}\right)^{0} = 2.
		      \]
		      Our base case is thence satisfied. Now, for the inductive step. Assume that
		      \[0 < \frac{2^n}{n!} \leq 2\left(\frac{2}{3}\right)^{n-2}\]
		      for some $n \geq 2$. Clearly, by the inductive hypothesis:
		      \begin{align*}
			      0 < \frac{2^{n+1}}{(n+1)!} = \frac{2}{(n+1)} \cdot \frac{2^n}{n!} & \leq \frac{2 \cdot 2}{(n+1)} \left(\frac{2}{3}\right)^{n-2} \\
			                                                                        & \leq 2 \left(\frac{2}{3}\right)^{n-2 + 1}                   \\
			                                                                        & =2 \left(\frac{2}{3}\right)^{(n + 1) - 2}.
		      \end{align*}
		      Here, we have used that $n\geq 2$ implies $n+1 \geq 3$ and, consequently, $\frac{1}{n+1} \leq \frac{1}{3}$. By our earlier remarks (where we discussed the application of the squeeze theorem), we see that \eqref{eq:lim4-1} holds.
	      \end{proof}
	\item 	Let $x_1 \geq 2$ be given and define for $n \geq 1$:
	      \begin{equation}\label{eq:4}
		      x_{n+1} := 1 + \sqrt{x_n - 1}.
	      \end{equation}
	      Prove that the recursive sequence $(x_n)$ converges and calculate its limit.
	      \begin{proof}
		      First, let us show (by induction) that $x_{n} \geq 2$ for all $n \in \NN$. The base case $n=1$ is trivial. Assuming that $x_n \geq 2$, we then have
		      \[
			      x_{n+1} = 1 + \sqrt{x_n - 1} \geq 1+ \sqrt{2-1} = 2.
		      \]
		      Here, we have used that the function $f(x) := \sqrt{x}$ is monotone increasing on $[0,\infty)$.\footnote{To see why this is true, suppose that $0 \leq x < y$. Then, we calculate
		      \begin{align*}
			      \sqrt{y} - \sqrt{x} = \frac{(\sqrt{y}-\sqrt{x})(\sqrt{y} + \sqrt{x})}{\sqrt{y} + \sqrt{x}}
			      = \frac{y - x}{\sqrt{x} + \sqrt{y}} \geq 0.
		      \end{align*}
		      Thus, $\sqrt{x} \leq \sqrt{y}$ whenever $0 \leq x < y$. Since this also holds for $0 \leq x= y$, we see that $f(x) = \sqrt{x}$ is monotone increasing on $[0,\infty)$.
		      } We conclude that $x_n \geq 2$ for all $n \in \NN$. In particular, $x_n - 1 \geq 1$ for all $n \in \NN$. Thus, given $n \in \NN$, we have
		      \[
			      x_{n+1} = 1 + \sqrt{x_n - 1} \leq 1 + (x_n - 1) = x_n.
		      \]
		      In this last step, we have used that $\sqrt{t} \leq t$ for all $t \geq 1$.\footnote{To justify this inequality, let $t \geq 1$ and note that \[
				      t - \sqrt{t} = \sqrt{t}(\sqrt{t}-1) \geq \sqrt{t} - 1 \geq 0.
			      \]}
		      Hence, $(x_n)$ is a monotone decreasing sequence that is bounded below by $2$. It therefore converges to some point $x \geq 2$. Using the limit laws and \eqref{eq:4}, this point $x$ must satisfy
		      \[
			      x = 1 + \sqrt{x-1}.
		      \]
		      It follows that $(x-1) = \sqrt{x-1}$, where $x-1 \geq 1$. Hence, $\sqrt{x-1}  = 1$ whence $x = 2$. We conclude that $\lim{x_n} = 2$.
	      \end{proof}
	\item Let $a > 0$ and define $x_1 := a$. For $n \geq 1$, let us put
	      \[
		      x_{n+1} := x_n + \frac{1}{x_n}.
	      \]
	      Prove that the sequence $(x_n)$ does not converge.
	      \begin{proof}
		      By induction, we will first show that $x_n \geq a > 0$, for all $n \in \NN$. The base case ($n=1$) is given to us. If $x_n \geq a$, then we automatically have\[
			      x_{n+1} = x_n + \frac{1}{x_n} > x_n \geq a.
		      \]
		      Thus, by the principle of induction, we see that $x_n \geq a$ for all $n \in \NN$. If $x_n$ converges to some point $x \in \RR$, we must then have $x \geq a$ (since $x_n \geq a$ for all $n \in \NN$). In particular, $x > 0$. Thus, by the limit laws, one has
		      \[
			      x = x + \frac{1}{x}.
		      \]
		      But this equation forces $0 = \frac{1}{x}$, which is a contradiction. Hence, $(x_n)$ cannot converge.
	      \end{proof}
	\item Let $(x_n)$ be a sequence in $\RR$ converging to $x \in \RR$. For $n \in \NN$, define a new sequence by setting
	      \[
		      y_n := \frac{x_1 + \cdots + x_n}{n}.
	      \]
	      Prove that $\lim{y_n} = \lim{x_n} = x$.
	      \begin{proof}
		      This one is tricky so let's try to be clear in our justification. Let $\epsilon > 0$ be given; we must show that there exists $N \in \NN$ with the property that
		      \begin{align}\label{eq-a}
			      \abs{ y_n - x } = \abs{ \frac{x_1 + \cdots + x_n}{n}  - x} = \abs{ \frac{(x_1-x) + \cdots + (x_n - x)}{n} } < \epsilon
		      \end{align}
		      for \emph{every} $n \geq N$. Now, since $\lim{x_n} = x$, we can find $K_1 \in \NN$ such that
		      \[
			      \abs{ x_n - x } < \frac{\epsilon}{2}, \quad \forall n \geq K_1.
		      \]
		      Using \eqref{eq-a}, we see that for all $n \geq K_1$:
		      \begin{align*}
			      \abs{ y_n - x } & \leq \abs{ \frac{(x_1-x) + \cdots + (x_n - x)}{n} }         \\
			                      & \leq \abs{\frac{ (x_1 - x) + \cdots + (x_{K_1} - x) }{n}}   \\
			                      & + \abs{\frac{ (x_{K_1 +1} - x)  + \cdots + (x_n - x) }{n}}.
		      \end{align*}
		      Note that the constant $$C := (x_1 - x) + \cdots + (x_{K_1} - x) $$ is independent of $n$. Returning to the above, we see that for all $n \geq K_1$:
		      \begin{align}\label{eq:b}
			      \abs{ y_n - x } & \leq \frac{\abs{C}}{n} + \abs{\frac{ (x_{K_1 +1} - x)  + \cdots + (x_n - x) }{n}}                  \\
			                      & \leq \frac{\abs{C}}{n} + {\frac{ \abs{x_{K_1 +1} - x}  + \cdots + \abs{x_n - x} }{n}}.\label{eq:c}
		      \end{align}
		      By the limit laws, it is obvious that
		      \[
			      \lim \frac{\abs{C}}{n} = 0.
		      \]
		      Hence, we can find $K_2 \in \NN$ such that
		      \[
			      \abs{ \frac{\abs{C}}{n}  - 0} = \frac{\abs{C}}{n} < \frac{\epsilon}{2}
		      \]
		      for all $n \geq K_2$. Define $N := \max(K_1,K_2) \in \NN$. Then, if $n \geq N$ we have both $n \geq K_1$ and $n \geq K_2$. Thus, for all $n \geq N$ \eqref{eq:c} shows that
		      \begin{align*}
			      \abs{ y_n - x } & \leq \frac{\abs{C}}{n} + {\frac{ \abs{x_{K_1 +1} - x}  + \cdots + \abs{x_n - x} }{n}}                                     \\
			                      & < \frac{\epsilon}{2}  + \frac{ \overbrace{\frac{\epsilon}{2} + \cdots + \frac{\epsilon}{2}}^{\text{($n-K_1$)-times} }}{n} \\
			                      & \leq \frac{\epsilon}{2} + \frac{n \cdot \frac{\epsilon}{2}}{n}                                                            \\
			                      & = \epsilon.
		      \end{align*}
	      \end{proof}
\end{enumerate}

\section*{Tutorial 9}

\begin{enumerate}[leftmargin=*]
	\item Suppose $D\subseteq \RR$ is a non-empty set and $f, g : D \to \RR$. Show that if $f(x) \leq g(y)$ for all $x, y\in D$ then
	      \[
		      \sup f(D) \leq \inf g(D)
	      \]
	      \begin{proof}
		      We first show that $f(D)$ is bounded above. Indeed, fix a point $y_0\in D$ and notice that $f(x) \leq g(y_0)$ for all $x\in D$. In particular, $z\leq g(y_0)$ for all $z\in f(D)$. We conclude that $f(D)$ is bounded above whence the supremum exists.
		      Similarly, show that $g(D)$ is bounded below.

		      Now, let $z$ be an arbitrary point in $f(D)$. Then $z = f(x_0)$ for some $x_0 \in D$. By assumption, $f(x_0) \leq g(y)$ for all $y\in D$. That is, $z = f(x_0)$ is a lower bound for $g(D)$. By definition of the infimum, it follows that $z \leq \inf g(D)$. Since this inequality holds for all $z\in f(D)$, this means that $\inf g(D)$ is an upper bound for $f(D)$. But then, by definition of the supremum, $\sup f(D) \leq \inf g(D)$.
	      \end{proof}

	\item Show that
	      \[
		      \frac{n!}{n^n} \to 0
	      \]
	      and
	      \[
		      \frac{n^2}{n!} \to 0
	      \]

	\item Suppose $x_n \geq 0$ converges to $x$. Show that $\sqrt{x_n} \to \sqrt{x}$.

	      \begin{proof} Two cases:

		      \textbf{Case 1}: $x = 0$. Given $\epsilon > 0$, let $N \in \NN$ be such that for all $n\geq N$ there holds
		      \[
			      \abs{x_n - 0} = x_n < \epsilon^2
		      \]
		      \[
			      \abs{\sqrt{x_n} -0} = \sqrt{x_n} < \sqrt{\epsilon^2} = \epsilon
		      \]

		      \textbf{Case 2}: $x>0$. Let $\epsilon > 0$ be given. First, we compute
		      \[
			      \abs{\sqrt{x_n} - \sqrt{x}}
			      =\frac{\abs{\sqrt{x_n} - \sqrt{x}}}{\sqrt{x_n} + \sqrt{x}}
			      \leq \frac{\abs{{x_n} - {x}}}{\sqrt{x}}
		      \]
		      So, let $N$ be sufficiently large so that $\abs{x_n - x} < \sqrt{x}\epsilon$ for all $n\geq N$.

		      Then for all $n\geq N$ we have
		      \[
			      \abs{\sqrt{x_n} - \sqrt{x}} \leq \frac{\abs{{x_n} - {x}}}{\sqrt{x}} < \epsilon.
		      \]
	      \end{proof}

	\item Archimedean property, Nested Intervals property, Monotone convergence theorem.

	\item Show that $\inf S = -\sup(-S)$.

\end{enumerate}

\section*{Tutorial 10}

\subsection*{Cauchy sequences}
This section is meant to make sense of Cauchy sequences, but it is not course material. So feel free to ignore what is written here if it only confuses you.\\

We wish to clarify why one bothers to define Cauchy sequences. In $\RR$, Cauchy sequences are just convergent sequences, so what's the point in introducing a new definition?

We begin by noting that there exists a sequence $(x_n) \subseteq \QQ$ which converges to an irrational number.
Now, suppose we do not yet have a concept of real (and in particular irrational) numbers. That is, suppose that we are familiar with the space $\QQ$ but have not defined $\RR$. Then what can we say about the sequence $(x_n)$? Well, it is certainly not convergent in $\QQ$! On the other hand, we can say that $x_n$ is Cauchy. In fact, one way of defining the real numbers is let the reals be all the ''limit points'' of Cauchy sequences in $\QQ$. Such a process is what a \emph{completion}. That is, we say that $\RR$ is the \emph{completion} of $\QQ$.

\begin{rem}
	We claimed above that there exists a sequence of rational numbers converging to an irrational number. For the sake of completeness, we show how to construct such a sequence. To this end, fix an irrational number $x$. For every $n\in\NN$ there exists (by the density of $\QQ$) a rational number $x_n$ such that
	\[
		x < x_n < x + \frac{1}{n}
	\]
	We then see by the Squeeze theorem that $(x_n) \to x$. Since $(x_n)$ is a sequence of rational numbers, this is what had to be shown.
\end{rem}


\subsection*{Exercises}

\begin{enumerate}[leftmargin=*]
	\item Bounded sequences are not necessarily convergent. e.g. $(-1)^n$. The Bolzano Weierstrass theorem guarantees the existence of a bounded Sub-sequence.

	\item $x_n \to 0$ and $y_n$ bounded then $x_ny_n \to 0$.

	\item Prove that
	      \[
		      \frac{(-1)^n}{n} + \frac{1}{n^2} - \cos\left(\pi n\right)
	      \]
	      diverges.
	      \begin{proof}
		      Suppose for a contradiction that the above sequence converges. Then since
		      \[
			      \frac{(-1)^n}{n} \quad\text{and}\quad\frac{1}{n^2}
		      \]
		      converge, we see by the limit laws that
		      \[
			      \frac{(-1)^n}{n} + \frac{1}{n^2} - \left(\frac{(-1)^n}{n} + \frac{1}{n^2} - \cos\left(\pi n\right)\right) = \cos(\pi n)
		      \]
		      converges. So, let $x$ denote the limit of $\cos(\pi n)$. Then any subsequence of $\cos(\pi n)$ converges to $x$. In particular
		      \[
			      \cos(\pi \cdot (2n)) = 1
			      \quad\text{and}\quad
			      \cos(\pi \cdot (2n+1)) = -1
		      \]
		      tend to $x$. So, $x=1$ and $x=-1$ which is absurd.
	      \end{proof}
	\item Let $(x_n)$ be an unbounded sequence. Show that there exists a subsequence $(x_{n_k})$ of $x_n$ such that
	      \[
		      \lim_{k\to\infty} \frac{1}{x_{n_k}} = 0
	      \]
	      \begin{proof}
		      Let $x_{n_1}$ be such that $\abs{x_{n_1}} > 1$ (note that we may indeed find such an element since $(x_n)$ is unbounded sequence). Now, for every $k\in \NN$, suppose that we have already defined $x_{n_k} > k$.

		      Suppose also, for a contradiction, that
		      \[
			      \abs{x_n} \leq k+1 \qquad\forall n > n_k
		      \]
		      Then, the sequence $(x_n)$ would be bounded above by
		      \[
			      \max\set{k+1, \abs{x_1}, \abs{x_2}, \dots, \abs{x_{n_k}}}
		      \]
		      Similarly, we see that $(x_n)$ is bounded below. But this contradicts the assumption that $(x_n)$ is unbounded. Therefore, we can find $x_{n_{k+1}}$ such that $\abs{x_{n_{k+1}}} > k+1$.

		      By this process, we have constructed a subsequence $(x_{n_k})$ of $(x_n)$ such that
		      \[
			      \abs{x_{n_k}} > k \quad\forall k\in\NN
		      \]
		      It remains to prove that
		      \[
			      \frac{1}{x_{n_k}} \to 0.
		      \]
		      For this, we begin by fixing $\epsilon > 0$. By the Archimedean property, there exists $K_\epsilon\in\NN$ such that $K_\epsilon > \frac{1}{\epsilon}$. Ergo, for all $k\geq K_\epsilon$ we see that
		      \[
			      \abs{\frac{1}{x_{n_k}} - 0} = \frac{1}{\abs{x_{n_k}}} < \frac{1}{k} \leq \frac{1}{K_\epsilon} < \epsilon
		      \]
		      This conclude the problem.
	      \end{proof}

	\item Let $(x_n)$ be a sequence of non-negative numbers and suppose that $\lim (-1)^n x_n$ exists. Then show that $(x_n)$ converges and determine it's limit.

	      \begin{proof}
		      Since $(-1)^nx_n$ converges to some $x\in \RR$, so do it's sub-sequences. In particular,
		      \[
			      (-1)^{2n}x_{2n} = x_{2n} \to x
			      \quad\text{and}\quad
			      (-1)^{2n+1}x_{2n+1} = -x_{2n+} \to x
		      \]
		      Since $(x_{2n})$ is a sequence of non-negative numbers, we se that $x\geq 0$. On the other hand, since $(-x_{2n+1})$ is a sequence of non-positive numbers, we also have $x \leq 0$. Combining the two inequalities, we see that $x=0$.

		      We are now ready to show that $x_n \to 0$. To this end, pick $\epsilon > 0$. Since $(-1)^nx_n \to x = 0$, there exists $n\in \NN$ such that
		      \[
			      \abs{(-1)^nx_n - 0} < \epsilon\qquad\forall n\geq N
		      \]
		      Therefore, we see that
		      \[
			      \abs{x_n -  0} = \abs{(-1)^nx_n - 0} < \epsilon \qquad\forall n\geq N
		      \]
		      By definition, this means that $x_n \to 0$.
	      \end{proof}

	      \iffalse
	\item Let $(x_n)$ be a sequence of real numbers and suppose there exists $r\in (0,1)$ such that
	      \[
		      \abs{x_{n+1} - x_{n}} < r^n
	      \]
	      for all $n\in\NN$. Show that the sequence $(x_n)$ converges.

	      \begin{proof}
		      It suffices to show that the sequence $(x_n)$ is Cauchy. Fix natural numbers $n > m$ and notice that
		      \begin{align*}
			      \abs{x_n - x_m}
			       & =\abs{x_n - x_{n-1} + x_{n-1} - \dots - x_{m+1} + x_{m+1} - x_m} \\
			       & \leq \sum_{k=0}^{n-m-1}\abs{x_{m+k+1} - x_{m+k}}                 \\
			       & < \sum_{k=0}^{n-m-1} r^{m+k}                                     \\
			       & =r^m \frac{1 - r^{n-m}}{1-r}
			      \leq \frac{1}{1-r}r^m
		      \end{align*}
		      Since $r^m \to 0$ as $m \to \infty$ and $\frac{1}{1-r}$ is constant, we know from class results that
		      \[
			      \frac{1}{1-r}r^m \to 0
		      \]
		      So, given $\epsilon > 0$, we may find $N \in \NN$ such that
		      \[
			      \frac{1}{1-r}r^m < \epsilon\qquad \forall n \geq N
		      \]
		      But then, our above inequalities show that for all $n, m \geq M$ there holds
		      \[
			      \abs{x_n - x_m} \leq \frac{1}{1-r}r^{\min\set{n, m}} < \epsilon.
		      \]
		      We conclude that $(x_n)$ is indeed Cauchy and therefore converges.
	      \end{proof}
	      \fi

	\item Show directly from the definition that
	      \[
		      \frac{n}{n+1}
	      \]
	      is a Cauchy sequence.

	      \begin{proof}
		      Given natural numbers $n, m$, notice that
		      \begin{align*}
			      \abs{\frac{n}{n+1} - \frac{m}{m+1}}
			       & =\abs{\frac{n(m+1) - m(n+1)}{(n+1)(m+1)}}     \\
			       & \leq \abs{\frac{n-m}{(n+1)(m+1)}}             \\
			       & \leq \frac{n+m}{(n+1)(m+1)}                   \\
			       & = \frac{n}{(n+1)(m+1)} + \frac{m}{(n+1)(m+1)} \\
			       & \leq \frac{1}{m+1} +\frac{1}{n+1}
		      \end{align*}
		      Now, fix $\epsilon > 0$ and let $N\in\NN$ be such that $N > \frac{2}{\epsilon}$. Then for all $n,m \geq N$ we have
		      \[
			      \abs{\frac{n}{n+1} - \frac{m}{m+1}} \leq \frac{1}{n+1} +\frac{1}{m+1} < \frac{1}{N} + \frac{1}{N} = \frac{2}{N} < \epsilon.
		      \]
	      \end{proof}

	\item Show directly from the definition that $(-1)^n$ is not a Cauchy sequence.

	      \begin{proof}
		      A sequence $(x_n)$ is said to be Cauchy if
		      \[
			      (\forall \epsilon > 0)(\exists N\in \NN)(\forall n, m \geq N)(\abs{x_n - x_m} < \epsilon).
		      \]
		      Therefore, a sequence in \textbf{not} Cauchy if
		      \[
			      (\exists \epsilon > 0)(\forall N\in \NN)(\exists n, m \geq N)(\abs{x_n - x_m} \geq \epsilon).
		      \]
		      We pick $\epsilon = 1$. For all $N\in \NN$, there exists $n, m \geq N$ such that $n$ is even and $m$ are odd. In this case, we have
		      \[
			      \abs{(-1)^n - (-1)^m} = \abs{1 - (-1)} = 2 \geq 1 = \epsilon.
		      \]
		      Therefore, $(-1)^n$ is not Cauchy.
	      \end{proof}

	\item Let $p$ be a fixed natural number. Establish a sequence $(x_n)$ that is not Cauchy but such that
	      \[
		      \lim_{n\to\infty}\abs{x_{n+p} - x_n} = 0.
	      \]
	      \begin{proof}
		      For every $n\in\NN$, define
		      \[
			      x_n = \sum_{k=1}^n \frac{1}{k}
		      \]
		      Then notice that
		      \[
			      \abs{x_{n+p} - x_n} = \sum_{k=1}^p \frac{1}{n + k} \leq \sum_{k=1}^p \frac{1}{n} = \frac{p}{n} \xrightarrow{n\to\infty} 0
		      \]
		      Therefore,
		      \[
			      \lim_{n\to\infty}\abs{x_{n+p} - x_n} = 0.
		      \]
		      It remains to show that $(x_n)$ is not Cauchy. To this end, fix $m, n \in \NN$ such that $n > m$. Then we see that
		      \[
			      \abs{x_n - x_m} = \sum_{k={m+1}}^n\frac{1}{k}
			      \geq \sum_{k={m+1}}^n\frac{1}{n}
			      = \frac{n-m}{n}
		      \]
		      So, for all $N\in \NN$, picking $m = N$ and $n = 2N$ yields
		      \[
			      \abs{x_n - x_m} \geq \frac{2N - N}{2N} = \frac{1}{2}
		      \]
		      Therefore, we see that $(x_n)$ is not Cauchy.
	      \end{proof}

	\item Let $x_1>0$ and define
	      \[
		      x_{n+1} = \frac{1}{2+x_n}
	      \]
	      for each $n\in\NN$.
	      \begin{enumerate}
		      \item Show that $x_n > 0$ for all $n\in\NN$,
		      \item Show that the sequence is contractive,
		      \item Determine the limit of the sequence.
	      \end{enumerate}

	      \begin{proof}
		      We prove (a) by induction. The base case $x_1>0$ is given. Now, if $x_n > 0$, then $x_n + 2 > 0$. Therefore,
		      \[
			      x_{n+1} = \frac{1}{2+x_n} > 0.
		      \]
		      We move on to part (b). Recall that the sequence is said to be contractive if there exists a constant $0<C<1$ such that
		      \[
			      \abs{x_{n+2} - x_{n+1}} \leq C\abs{x_{n+1} - x_n}
		      \]
		      for all $n\in \NN$. To prove the above inequality, fix $n\in \NN$ and notice that
		      \begin{align*}
			      \abs{x_{n+2} - x_{n+1}}
			      =\abs{\frac{1}{2+x_{n+1}} - \frac{1}{2+x_n}}
			       & =\abs{\frac{2+x_n - (2+x_{n+1})}{(2+x_n)(2+x_{n+1})}}              \\
			       & = \abs{x_{n+1} - x_n}\cdot \frac{1}{2+x_n}\cdot\frac{1}{2+x_{n+1}} \\
			       & \leq \frac{1}{4}\abs{x_{n+1} - x_n}
		      \end{align*}
		      Therefore, we see that $(x_n)$ is a contractive sequence with $1/4$ as the constant of the contractive sequence.

		      (c) Since contractive sequences converge, we deduce that $x_n \to x$ for some $x\in \RR$. Recalling that
		      \[
			      x_{n+1} = \frac{1}{2+x_n},
			      \quad\text{and}\quad
			      x_n > 0
		      \]
		      for all $n\in\NN$, we may apply the limit laws (Theorem 3.23) in the textbook to deduce that
		      \[
			      x = \frac{1}{2+x} \implies x^2 + 2x -1 = 0
		      \]
		      Therefore, the only possible values for $x$ are $\sqrt{2} - 1$ and $-\sqrt{2} - 1$. Finally, since $x_n >0$ for all $n\in\NN$, we see that $x$ must be equal to $\sqrt{2} - 1$.
	      \end{proof}

	\item Let $x_1 < x_2$ and for each $n \in \NN$, define
	      \[
		      x_{n+2} = \frac{x_n + x_{n+1}}{2}
	      \]
	      Does this sequence converge? If so, what is it's limit.

	      \begin{proof}
		      To see that the sequence converges, we will show that it is contractive. Indeed, fix $n\in \NN$ and notice that
		      \[
			      \abs{x_{n+2} - x_{n+1}} = \abs{\frac{x_n + x_{n+1}}{2} - x_{n+1}} = \frac{1}{2} \abs{x_{n+1} - x_n}.
		      \]
		      So, $(x_n)$ is converges to some $x\in \RR$. Now, for all $n\in \NN$ there holds
		      \[
			      2x_{n+2} + x_{n+1} = 2\frac{x_n + x_{n+1}}{2} + x_{n+1} = 2x_{n+1} + x_n
		      \]
		      Therefore, we see that
		      \[
			      2x_{n+1} + x_n = 2x_2+x_1
		      \]
		      Applying the appropriate limit laws, we conclude that
		      \[
			      3x = 2{x_2} + x_1
			      \implies x = \frac{2x_2 + x_1}{3}.
		      \]
	      \end{proof}

	\item We present a similar exercise. Let $x_1 < x_2$ and for each $n \in \NN$, define
	      \[
		      x_{n+2} = \frac{2x_n + x_{n+1}}{3}
	      \]
	      Does this sequence converge? If so, what is it's limit.

	      \begin{proof}
		      As in the previous exercise, we will show that the sequence converges by proving that it is contractive. To this end, fix $n\in \NN$ and notice that
		      \[
			      \abs{x_{n+2} - x_{n+1}} = \abs{\frac{2x_n + x_{n+1}}{3} - x_{n+1}} = \frac{2}{3} \abs{x_{n+1} - x_n}.
		      \]
		      So, $(x_n)$ is converges to some $x\in \RR$. Now, for all $n\in \NN$ there holds
		      \[
			      3x_{n+2} + 2x_{n+1} = 2x_n + x_{n+1} + 2x_{n+1} = 3x_{n+1} + 2x_n
		      \]
		      Therefore, we see that
		      \[
			      3x_{n+1} + 2x_n = 3x_2+2x_1
		      \]
		      We conclude that
		      \[
			      5x = 3{x_2} + 2x_1
			      \implies x = \frac{3x_2 + 2x_1}{5}.
		      \]
	      \end{proof}

\end{enumerate}

\section*{Tutorial 11}

\subsection*{Midterm Review}

\subsubsection*{Exercise 2} Given $a>0$, suppose $S \subseteq (a, \infty)$ is a non-empty set and define
\[
	T = \set{\frac{1}{s} :  s\in S}
\]
Justify why $\sup T$ and $\inf S$ exists. Show that $\sup T = 1/\inf(S)$.

\begin{proof} I will prove the statement directly from the definition.

	Clearly, $S$ is bounded below by $a$ and $T$ is bounded above by $1/a$. Therefore, the infimum of $S$ and the supremum of $T$ exist. For the second part of the problem, let $u:=\inf(S)$. We first show that $1/u$ is an upper bound of $T$. Indeed, any element $t\in T$ can be written as $t = 1/s$ for some $s\in S$.  Since $u$ is a lower bound for $S$, we know that $s \geq u$. It follows that
	\[
		t = \frac{1}{s} \leq \frac{1}{u}
	\]
	Since $t$ was an arbitrary element of $T$, we conclude that $1/u$ is indeed an upper bound for $T$. It remains to show that $1/u$ is the least upper bound. For this, suppose $v$ is another upper bound for $T$. Then we claim that $v > 0$ and that $1/v$ is a lower bound for $S$. It is clear that $v>0$. Now, for any $s\in S$, the element $1/s$ is in $T$. Therefore, $1/s\leq v$.  It follows that $s \geq 1/v$. Since $s$ was an arbitrary element of $S$, it follows that $1/v$ is a lower bound for $S$. Since $u$ is the greatest lower bound for $S$, there holds that
	\[
		\frac{1}{v} \leq u \implies v \geq \frac{1}{u}
	\]
	So, we have shown that any other upper bound for $T$ is greater or equal to $1/u$. Thus, $1/u$ is the least lower bound of $T$.
\end{proof}

\subsubsection*{Exercise 3} Show that if $(y_n)$ is a bounded sequence and $x_n \to 0$ then the sequence $y_nx_n$ also converges to 0.
\begin{proof}
	If $y_n$ is bounded, then there exists some $M>0$ such that $\abs{y_n} \leq M$ for all $n\in \NN$. Now, given $\epsilon>0$ there exists $N\in \NN$ such that $\abs{x_n} < \frac{\epsilon}{M}$ for all $n\geq N$. It follows that
	\[
		\abs{y_nx_n} = \abs{y_n}\cdot\abs{x_n} \leq M \abs{x_n} < M\frac{\epsilon}{M} = \epsilon
	\]
	for all $n\geq N$. By definition, this means that $y_nx_n \to 0$.
\end{proof}

The above statement is not necessarily true if $y_n$ is not bounded. Consider for instance the sequences $x_n = \frac{1}{n}$ and $y_n = -n$.

\subsubsection*{Exercise 5} (Bonus Question) Let $y_1 = 2$ and $y_{n+1} = y_n +\frac{n}{y_n}$ for all $n\geq 1$. Does the sequence $(y_n)$ converge?

\begin{proof}
	First, notice that $y_n \geq 2$ for all $n\in\NN$ hence the sequence is well-defined. Suppose now that $(y_n)$ converges. For each $n\in\NN$, there holds
	\[
		y_{n+1}y_n - y_n^2 = n
	\]
	Since $(y_n)$ converges, we see that the left hand side of the above is a convergent sequence. Therefore, the sequence $(n)$ converges. Since convergent sequences are bounded, there exists $x \in \RR$ such that
	\[
		n \leq x
	\]
	for all $n\in \NN$. But this contradicts the Archimedean property. Therefore, $(y_n)$ cannot converge.
\end{proof}
\begin{rem}
	Actually, one can show that $(y_n)$ tends to positive infinity. To see this, notice first that $(y_n)$ is an increasing sequence. If it were bounded above, then the monotone convergence theorem would imply that $(y_n)$ converges. But as we have shown above, $(y_n)$ does not converge. We deduce that the sequence $(y_n)$ is increasing and unbounded. Ergo, $y_n \to +\infty$.
\end{rem}

\subsection*{Exercises}

\begin{enumerate}[leftmargin=*]
	\item Show that the follows sequences are properly divergent
	      \[
		      \frac{n^2-2}{n+2}
		      \quad\text{and}\quad
		      n + 2 - n^2
	      \]
	      \begin{proof}
		      For the first sequence, we first notice that
		      \[
			      \frac{n^2-2}{n+2} \geq \frac{n^2-4}{n+2} = \frac{(n-2)(n+2)}{n+2} = n-2
		      \]
		      So, for any $a\in \RR$, we may pick (by the Archimedean property) $N\in \NN$ such that $N > a+2$. Then for all $n\geq N$ there holds
		      \[
			      \frac{n^2-2}{n+2} = n - 2 \geq N-2 > (a+2)-2 = a
		      \]
		      By definition, this implies that
		      \[
			      \frac{n^2-2}{n+2} \to + \infty
		      \]
		      Now, for the second sequence we first notice that
		      \[
			      n + 2 - n^2 = n(1-n) + 2 \leq 1-n+2 = 3-n
		      \]
		      for all $n\in \NN$. Indeed, the above inequality is true since
		      \[
			      n(1-n) \leq 1-n \iff n-n^2 \leq 1-n \iff 0 \leq n^2-2n+1 = (n-1)^2
		      \]
		      So, given $a\in \RR$ we may pick $N\in \NN$ such that $N > a+3$. Then for all $n\geq N$ we have
		      \[
			      n + 2 - n^2 \leq 3 - n < 3 - (a+3) = a
		      \]
		      Once again, it follows from the definition that $n+2-n^2 \to - \infty$.
	      \end{proof}

	\item Show that the sequence
	      \[
		      x_n = \sum_{k=1}^{n}\frac{1}{k^2}
	      \]
	      converges.

	      \begin{proof}
		      By the monotone convergence theorem, it suffices to show that $(x_n)$ is increasing and bounded above. First, it is clear that $(x_n)$ is an increasing sequence. To see that $(x_n)$ is bounded above, we fix $n\in \NN$ an proceed as follows;
		      \begin{align*}
			      x_n = \sum_{k=1}^{n}\frac{1}{k^2}
			      = 1 + \sum_{k=2}^{n}\frac{1}{k^2}
			       & \leq 1 + \sum_{k=2}^{n}\frac{1}{k(k-1)}                      \\
			       & = 1 + \sum_{k=2}^{n}\left[\frac{1}{k-1} - \frac{1}{k}\right] \\
			       & = 1+ \left[1 - \frac{1}{n}\right] \leq 2
		      \end{align*}
		      This concludes the proof.
	      \end{proof}

	\item Suppose $(x_n)$ be a sequence of positive numbers (i.e. $x_n>0$ for all $n\in \NN$). Show that $x_n \to +\infty$ if and only if $\frac{1}{x_n} \to 0$.
	      \begin{proof}
		      Suppose first that $x_n \to +\infty$. Fix $\epsilon > 0$ and notice that, by definition, there exists $N\in \NN$ such that
		      \[
			      x_n > \frac{1}{\epsilon} \qquad \forall n\geq N
		      \]
		      It follows that
		      \[
			      \abs{\frac{1}{x_n} - 0} = \frac{1}{x_n} < \epsilon
		      \]
		      for all $n\geq \NN$. That is, $x_n \to 0$.

		      Conversely, suppose that $1/x_n \to 0$ and let $a\in \RR$ be given. If $a\leq 0$, then $x_n \geq 0 > a$ for all $n \geq 1$. Otherwise, we may pick $N\in \NN$ such that
		      \[
			      \frac{1}{x_n} = \abs{\frac{1}{x_n} - 0} < \frac{1}{a}
		      \]
		      for all $n\geq N$. It follows that
		      \[
			      x_n > a
		      \]
		      for all $n\geq N$. Since the choice of $a\in \RR$ was arbitrary, we see that $x_n \to +\infty$.
	      \end{proof}

	\item Let $(x_n)$ and $(y_n)$ be positive sequences. Suppose that $x_n \to +\infty$ and $(x_ny_n)$ is bounded. Then show that $y_n \to 0$.

	      \begin{proof}
		      Since $(x_ny_n)$ is positive bounded, there exists $M>0$ such that
		      \[
			      0 < x_ny_n < M
		      \]
		      for all $n\in \NN$. Now, fix $\epsilon > 0$ and let $N\in \NN$ be such that
		      \[
			      x_n > \frac{M}{\epsilon}\iff \frac{1}{x_n} < \frac{\epsilon}{M}
		      \]
		      for all $n\geq N$. Then
		      \[
			      \abs{y_n - 0} = y_n = y_n x_n\cdot\frac{1}{x_n} < M\cdot\frac{\epsilon}{M} = \epsilon
		      \]
		      By definition, this means that $y_n \to 0$.
	      \end{proof}

	\item Suppose $f:\RR\to\RR$ is a function and $c\in \RR$. Show that $\lim_{x\to c}f(x) = L$ if and only if $\lim_{x\to 0} f(x+c) = L$.

	      \begin{proof}
		      Suppose first that $\lim_{x\to c}f(x) = L$. The for all $\epsilon > 0$, there exists $\delta>0$ such that
		      \[
			      \abs{f(x) - L} < \epsilon
		      \]
		      whenever $0<\abs{x-c} < \delta$. But then, whenever $0<\abs{x-0} < \delta$ we see that $0<\abs{(x+c) - c} < \delta$. By the above,
		      \[
			      \abs{f(x+c) - L} <\epsilon
		      \]
		      That is, $\lim_{x\to 0} f(x+c) = L$. We leave the converse as an exercise.
	      \end{proof}

	\item Show directly from the definition that $\lim_{x\to c}x^2 = c^2$. Show also how to prove this by using the sequential criterion.

	      \begin{proof}
		      We first notice that, for all $x$ such that $\abs{x-c}\leq1$, there holds
		      \begin{align*}
			      \abs{x^2 - c^2} = \abs{x-c}\abs{x+c}
			       & =\abs{x-c}\left(\abs{x-c+2c}\right)               \\
			       & \leq \abs{x-c}\left(\abs{x - c} + \abs{2c}\right) \\
			       & \leq \abs{x-c}\left(1 + 2\abs{c}\right)
		      \end{align*}
		      So, given $\epsilon > 0$ let
		      \[
			      \delta = \min\set{1, \frac{\epsilon}{1+2\abs{c}}}
		      \]
		      Then for all $x$ satisfying $0<\abs{x-c} < \delta$ we see that
		      \[
			      \abs{x^2 - c^2} \leq \abs{x-c}\left(1 + 2\abs{c}\right) < \delta\left(1 + 2\abs{c}\right) \leq \epsilon
		      \]
		      By definition, this means that $x^2 \to c^2$.

		      We know prove the same result with the sequential criterion. To this end, let $(x_n)$ be an arbitrary sequence converging to $c$. By class results, we know that $x_n^2 = x_n \cdot x_n \to c^2$. This concludes the proof.
	      \end{proof}

	\item Use the definition of the limit to show that
	      \[
		      \lim_{x\to 6}\frac{x^2-3x}{x+3} = 2.
	      \]

	      \begin{proof}
		      We begin with some computations;
		      \begin{align*}
			      \abs{\frac{x^2-3x}{x+3} - 2}
			      =\abs{\frac{x^2-3x - 2(x+3)}{x+3}}
			       & =\abs{\frac{x^2-5x-6}{x+3}}     \\
			       & =\abs{\frac{(x-6)(x+1)}{x+3}}   \\
			       & =\abs{x-6}\abs{\frac{x+1}{x+3}}
		      \end{align*}
		      Now, notice that whenever $\abs{x-6} < 1$, we have
		      \[
			      \abs{x+1} = \abs{x-6 + 7} \leq \abs{x-6} + 7 \leq 8
		      \]
		      and
		      \[
			      \abs{x+3} = \abs{9+x-6}\geq 9 -\abs{x-6} \geq 8
		      \]
		      Combining the above inequalities we see that if $\abs{x-6} < 1$ then
		      \[
			      \abs{x-6}\frac{x+1}{x+3} \leq \abs{x-6}
		      \]
		      So, given $\epsilon>0$ let
		      \[
			      \delta = \min\set{\epsilon, 1}
		      \]
		      Then whenever $0<\abs{x-6}<\delta$ we see that
		      \[
			      \abs{\frac{x^2-3x}{x+3} - 2} \leq \abs{x-6} < \delta \leq \epsilon
		      \]
		      This concludes the proof.
	      \end{proof}

	\item Show that the following limits do not exist:
	      \[
		      \lim_{x\to 0} \frac{1}{x^2}
		      \quad\text{and}\quad
		      \lim_{x\to 0} x+\operatorname{sgn}(x)
	      \]
	      \begin{proof}
		      Suppose for a contradiction that
		      \[
			      \lim_{x\to 0} \frac{1}{x^2}
		      \]
		      exists. Then for any sequence $(x_n)$ such that $x_n \to 0$ and $x_n\neq 0$ for all $n\in \NN$, $\frac{1}{x_n^2}$ also converges. In particular, since $1/n \to 0$, the sequence
		      \[
			      \frac{1}{(1/n)^2} = n^2
		      \]
		      converges. But this is impossible since convergent sequences are bounded while $(n^2)$ is unbounded.

		      Similarly, suppose for a contradiction that
		      \[
			      \lim_{x\to 0} x+\operatorname{sgn}(x)
		      \]
		      exists and is equal to $L$. Then since $1/n \to 0$, we must have that
		      \[
			      L = \lim_{n\to\infty}\left[\left(\frac{1}{n}\right) + \operatorname{sgn}\left(\frac{1}{n}\right)\right]
			      = \lim_{n\to\infty} \frac{1}{n} + 1 = 1
		      \]
		      On the other hand, since $-1/n \to 0$ we also have
		      \[
			      L = \lim_{n\to\infty}\left[\left(-\frac{1}{n}\right) + \operatorname{sgn}\left(-\frac{1}{n}\right)\right]
			      = \lim_{n\to\infty} \frac{1}{n} - 1 = -1.
		      \]
	      \end{proof}

	\item Suppose $f:\RR\to\RR$ is a function such that $\lim_{x\to 0} f(x) = L$. Show that if $g(x): = f(ax)$ for some constant $a>0$ then $\lim_{x\to 0} g(x) = L$.

	      \begin{proof}
		      Fix $\epsilon>0$ and let $\delta>0$ be such that
		      \[
			      \abs{f(x) - L} < \epsilon
		      \]
		      whenever $0<\abs{x} < \delta$. But then for all $x$ such that $0 < \abs{x} < \delta/a$ we have
		      \[
			      \abs{g(x) - L} = \abs{f(ax) - L} < \epsilon
		      \]
		      where the last inequality holds since
		      \[
			      0 < \abs{ax} = a\abs{x} < a\frac{\delta}{a} = \delta.
		      \]
	      \end{proof}

\end{enumerate}

\section*{Tutorial 12}

\begin{enumerate}[leftmargin=*]
	\item Let $A$ be a subset of the reals and suppose that $c$ is a cluster point of $A$. Show that $-c$ is a cluster point of
	      \[
		      -A = \set{-a \mid a\in A}
	      \]
	      If $0\not\in A$ and $c\neq 0$, then is $1/c$ a cluster point of
	      \[
		      B = \set{\frac{1}{a} \mid a\in A}
	      \]
	      \begin{proof}
		      The first part follows immediately from the definition. Indeed, for any $\delta>0$, there exists a point $x\in A\setminus\set{c}$ such that $\abs{x-c} < \delta$. It then follows that the point $-x \neq -c$ is in $-A$ and
		      \[
			      \abs{-x-(-c)} = \abs{x-c}  <\delta
		      \]
		      For the second part of the problem, we prove that $1/c$ is a cluster point of $B$. To this end, fix $\delta>0$. By assumption, for any $\delta^\prime>0$ we may find $x\in A, x\neq c$ such that $\abs{x-c} < \delta^\prime$. If $x$ is as such then
		      \begin{equation}\label{eq:1}
			      \abs{\frac{1}{x} - \frac{1}{c}} = \abs{\frac{x-c}{xc}} <\frac{\delta^\prime}{\abs{xc}}
		      \end{equation}
		      Since $c \neq 0$, we know that $\abs{c} > 0$. Furthermore, if $\delta^\prime \leq \abs{c}/2$ then
		      \[
			      \abs{x} \geq \abs{c} -  \abs{x-c} \geq \abs{c} - \delta^\prime \geq \frac{\abs{c}}{2}
		      \]
		      In particular, equation \eqref{eq:1} may be bounded by
		      \[
			      \abs{\frac{1}{x} - \frac{1}{c}} <\frac{\delta^\prime}{\abs{x}\cdot\abs{c}} \leq \frac{\delta^\prime}{\abs{c}^2/2}
		      \]
		      Finally, if we pick
		      \[
			      \delta^\prime = \min\set{\frac{\abs{c}}{2}, \frac{\abs{c}^2}{2}\delta}
		      \]
		      Then there exists $x\in A, x\neq c$ such that $\abs{x-c} < \delta$ whence
		      \[
			      \abs{\frac{1}{x} - \frac{1}{c}} < \frac{\delta^\prime}{\abs{c}^2/2} < \delta.
		      \]
		      Since $1/x$ is an element of $B$ not equal to $1/c$, we are done.
	      \end{proof}

	\item Let $f, g$ be defined on a subset $A\subseteq \RR$ and suppose $c$ is a cluster point of $A$. Suppose $f$ is bounded on a neighbourhood of $c$ and $\lim_{x\to c} g(x) = 0$. Show that
	      \[
		      \lim_{x\to c} (fg)(x) = 0
	      \]
	      \begin{proof}
		      Since $f$ is bounded on a neighbourhood of $c$, there exist $\eta>0$ and $M > 0$ such that for all $x\in A$ with $\abs{x-c} < \eta$ there holds $\abs{f(x)} \leq M$. Given $\epsilon > 0$, since $\lim_{x\to c} g(x) = 0$, there exists $\delta > 0$ such that for all $x\in A$ satisfying $0<\abs{x-c} < \delta$ there holds
		      \[
			      \abs{g(x)} = \abs{g(x) - 0} < \frac{\epsilon}{M}
		      \]
		      But then, for all $x\in A$ satisfying $0 < \abs{x-c} < \min\set{\delta, \eta}$ there holds
		      \[
			      \abs{f(x)g(x)} = \abs{f(x)} \cdot\abs{g(x)} \leq M\abs{g(x)} < M\frac{\epsilon}{M} = \epsilon
		      \]
		      By definition of the limit, we conclude that $\lim_{x\to c} (fg)(x) = 0$.
	      \end{proof}

	\item Determine the limit
	      \[
		      \lim_{x\to 2}\abs{\frac{x^2 - 4}{x-2}}\qquad (x>0)
	      \]
	      \begin{proof}
		      We show that the limit is $4$. For this, fix $\epsilon > 0$ and let $\delta = \epsilon$. Then for all $x > 0$ such that $0<\abs{x-2} < \delta$, we have
		      \[
			      \abs{\frac{x^2-4}{x-2} - 4} = \abs{x+2 - 4} = \abs{x-2} < \delta = \epsilon.
		      \]
	      \end{proof}

	\item Find examples of function $f, g :\RR\to \RR$ which do not have a limit at a point $c$ but such that $f+g$ and $fg$ have a limit at $c$.

	      \begin{proof}
		      Define
		      \[
			      f(x) = \begin{cases}
				      1  & x\in \QQ     \\
				      -1 & x\not\in \QQ
			      \end{cases}
		      \]
		      and
		      \[
			      g(x) = \begin{cases}
				      -1 & x\in \QQ      \\
				      1  & x\not\in \QQ.
			      \end{cases}
		      \]
		      One can show that $f$ and $g$ do not have a limit at any point $c\in \RR$. On the other hand, $f+g \equiv 0$ and $fg \equiv -1$ have well-defined limits at all points in $\RR$.
	      \end{proof}

	\item Suppose $f:\RR\to \RR$ is such that $f(x+y) = f(x) + f(y)$ for all $x,y\in \RR$. Show that if $\lim_{x\to 9}f(x)$ exists then
	      \[
		      \lim_{x\to c} f(x) = f(c)
	      \]
	      for every point $c\in \RR$.\footnote{This implies that the function is everywhere continuous.}

	      \begin{proof}
		      By assumption, we know that
		      \[
			      \lim_{x\to 0}f(x) = L.
		      \]
		      for some $L \in \RR$. Then
		      \[
			      2L = \lim_{x\to 0}\left(f(x) + f(x)\right) = \lim_{x\to 0} f(2x) = L.
		      \]
		      Notice that the last equality follows from problem 9 on the last tutorial. So, we have found that $2L = L$ which implies that $L = 0$.

		      Now, given $c\in \RR$, we know (by problem 5 in the last tutorial) that
		      \[
			      0 = \lim_{x\to 0} f(x) = \lim_{x\to c} f(x-c).
		      \]
		      Furthermore, since $f(c)$ is a constant function there holds
		      \[
			      \lim_{x\to c} f(c) = f(c).
		      \]
		      We conclude that the limit as $x$ tends to $c$ of $f(x-c) + f(c)$ exists and is equal to $f(c)$. Ergo,
		      \[
			      f(x) = \lim_{x\to c} \left[f(x-c) + f(c)\right] = \lim_{x\to c} f(x).
		      \]
	      \end{proof}


	\item Fix $a>0$ and let $f > 0$ be defined on $(a, \infty)$. Given $c \geq a$, show that $\lim_{x\to c}f(x) = \infty$ if and only if $\lim_{x\to c} 1/f(x) = 0$.

	      \begin{proof}
		      This result easily follows from the Sequential criterion for the limit. Suppose first that $\lim_{x\to c}f(x) = \infty$. Then for any sequence $(x_n) \subseteq (a, \infty)\setminus\set{c}$ such that $x_n \to c$, the sequence $f(x_n)$ tends to positive infinity. From results in a previous tutorial, it follows that
		      \[
			      \lim_{n\to\infty}\frac{1}{f(x_n)} = 0.
		      \]
		      We conclude that $\lim_{x\to c} 1/f(x) = 0$. The converse is handled similarly.
	      \end{proof}
\end{enumerate}

\section*{Tutorial 13}

\subsection*{Thomae's Function} We present here an example of a function which might defy our initial intuition regrading continuity. Thomae's function is given by
\[
	f(x) = \begin{cases}
		1           & x=0                                                                     \\
		\frac{1}{q} & \text{if } x= \frac{p}{q} \text{ for some coprime } p \in \ZZ, q\in \NN \\
		0           & \text{otherwise}
	\end{cases}
\]
We sill show that this function is discontinuous at all rational points but continuous at all irrational points.

We begin by proving that $f$ is discontinuous at all rational points. To this end, fix a point $c\in \QQ$ and write
\[
	c = \frac{p}{q}
\]
for some $p\in \ZZ$ and $q\in \NN$ such that $p$ and $q$ are co-prime (if $c=0$ then we write $c=0/1$). To show that $f$ is discontinuous at $c$, we have to show that
\[
	(\exists \epsilon > 0)(\forall \delta > 0)(\exists x\text{ such that } \abs{x-c} < \delta \text{ and } \abs{f(x) - f(c)} \geq \epsilon).
\]
We claim that $\epsilon = \frac{1}{q}$ will do. Indeed, for any $\delta > 0$ there exists (by density of the irrationals) a number $x\not\in\QQ$ such that
\[
	c < x < c+\delta
\]
Then $\abs{x-c} < \delta$ and
\[
	\abs{f(x) - f(c)} = \abs{0 - \frac{1}{q}} = \frac{1}{q} \geq \epsilon.
\]

We now prove that the function is continuous at any given rational.
To this end, pick an arbitrary point $c\in\RR\setminus\QQ$ and fix $\epsilon>0$. By the Archimedean property, there exists $n\in \NN$ such that
\[
	n > \frac{1}{\epsilon} \iff \frac{1}{n} < \epsilon
\]
Furthermore, for each $1\leq k\leq n$ there exists a natural number $m_k$ such that
\[
	\frac{m_k}{k} < c < \frac{m_k+1}{k}
\]
Indeed, we pick $m_k = \lfloor ck \rfloor$. Now, we define
\[
	\delta = \min_{1\leq k \leq n}\set{\abs{\frac{m_k}{k} - c}, \abs{\frac{m_k+1}{k} - c}}
\]
Finally, suppose that $\abs{x-c} < \delta$. If $x\not\in\QQ$ then
\[
	\abs{f(x) - f(c)} = \abs{0} = 0
\]
On the other hand, if $x\in \QQ$ then we may write
\[
	x = \frac{p}{q}
\]
for some $p\in\ZZ$ and $q\in\NN$ such that $p$ and $q$ are co-prime.

We claim that $q\geq n$ whence
\[
	\abs{f(x) - f(c)} = \frac{1}{q} \leq \frac{1}{n} < \epsilon
\]
in which case we would be done.

Suppose for a contradiction that $q<n$. Then one of the following must hold;
\[
	\frac{p}{q} \leq \frac{m_q}{q}
	\quad\text{or}\quad
	\frac{p}{q} \geq \frac{m_q+1}{q}.
\]
In either case, we see that
\[
	\abs{x-c}
	= \abs{\frac{p}{q} - x}
	\geq \min\set{\abs{\frac{m_q}{q} - c}, \abs{\frac{m_q+1}{q} - c}} \geq \delta
\]
which contradicts the assumption that $\abs{x-c} < \delta$.

\subsection*{Exercises}

\begin{enumerate}[leftmargin=*]
	\item Let $a<b<c$ and suppose $f$ is continuous on $[a,b]$ and $g$ is continuous on $[b,c]$. If $f(b) = g(b)$, then is the function
	      \[
		      h(x) = \begin{cases}
			      f(x) & x\in [a,b] \\
			      g(x) & x\in [b,c]
		      \end{cases}
	      \]
	      continuous on $[a,c]$.
	      \begin{proof}
		      We show that $h$ is indeed continuous. To this end, pick an arbitrary point in $p \in [a,c]$ and fix $\epsilon > 0$. If $a \leq p < b$, then we know that there exists $\delta > 0$ such that if $x\in [a,b]$ satisfies $\abs{x-p} < \delta$ then
		      \[
			      \abs{h(x) - h(p)} = \abs{f(x) - f(p)} < \epsilon
		      \]
		      Therefore, for all $x\in [a,c]$ such that $\abs{x-p} < \min\set{\delta, b-p}$ we have $\abs{h(x) - h(p)} < \epsilon$. We deduce that if $a\leq p < b$ the $h$ is continuous at $p$. A similar argument shows that $h$ is continuous on $(b, c]$. It remains to prove that $h$ is continuous at $b$. To this end, let $\delta_1$ be such that
		      \[
			      \abs{f(x) - f(b)} < \epsilon
		      \]
		      whenever $x\in [a,b]$ satisfies $\abs{x-b} < \delta_1$. Similarly, let $\delta_2$ be such that
		      \[
			      \abs{g(x) - g(b)} < \epsilon
		      \]
		      whenever $x\in [b,c]$ satisfies $\abs{x-b} < \delta_2$. Finally, define $\delta = \min\set{\delta_1, \delta_2}$ and notice that whenever $x\in [a,c]$ satisfies $\abs{x-b} < \delta$ there holds
		      \begin{align*}
			      \abs{h(x) - h(b)} & = \begin{cases}
				                            \abs{f(x) - f(b)} & x\in [a,b] \\
				                            \abs{g(x) - g(b)} & x\in [b,c]
			                            \end{cases} \\
			                        & <\epsilon.
		      \end{align*}
	      \end{proof}

	\item Let $A\subseteq\RR$ and suppose $f:A\to\RR$ is continuous at a point $c\in A$. Show that for all $\epsilon > 0$ there exists $\delta > 0$ such that for all $x,y\in A\cap V_\delta(c)$ there holds $\abs{f(x) - f(y)} < \epsilon$.

	      \begin{proof}
		      Fix $\epsilon > 0$ and let ${\delta}$ be such that $\abs{f(x) - f(c)} < \epsilon/2$ for all $x\in A$ such that $\abs{x-c} < {\delta}$. Now, suppose $x,y\in A\cap V_\delta(c)$ and notice that
		      \[
			      \abs{f(x) - f(y)} \leq \abs{f(x) - f(c)} + \abs{f(c) - f(y)} \leq \frac{\epsilon}{2} + \frac{\epsilon}{2} = \epsilon.
		      \]
	      \end{proof}

	\item Suppose $f:\RR\to\RR$ is continuous at a point $c\in A$. Show that if $f(c) > 0$ then there exists a neighbourhood of $c$ on which $f$ is strictly positive. In particular, the set $\set{x \in \RR : f(x) > 0}$ is open.

	      \begin{proof}
		      Pick $\epsilon := {f(c)}/2$ and let $\delta > 0$ be such that $\abs{f(x) - f(c)} < \epsilon$ for all $x$ such that $\abs{x-c} < \delta$. Then for any $x$ in the neighbourhood $V_\delta(c)$, we see that
		      \[
			      \abs{f(x)} \geq \abs{f(c)} - \abs{f(x) - f(c)} > \frac{f(c)}{2} > 0.
		      \]
	      \end{proof}

	      \begin{rem}
		      Similarly, one can show that $\set{x \in \RR :  f(x) < 0}$ is open. Notice that it immediately follows that $\set{x\in \RR : f(x) = 0}$ is closed.
	      \end{rem}

	\item Suppose $A\subseteq B\subseteq \RR$ and let $g:B\to \RR$ be a function. Suppose furthermore that $f$ is the restriction of $g$ to the set $A$. If $g$ is continuous at a point $a\in A$, then so is $f$. Show by way of counter-example that the converse is not necessarily true.

	      \begin{proof}
		      We show only that the converse might not hold. To this end, define $A = [0,1]$ and $B = [-1,1]$. Consider furthermore the function
		      \[
			      g(x) = \begin{cases}
				      0 & -1 \leq x < 0   \\
				      1 & 0 \leq x \leq 1
			      \end{cases}
		      \]
		      The notice that the function $f$ which is the restriction of $g$ to $A=[0,1]$ is continuous at $0$. On the other hand, $g$ is clearly not continuous at $0$.
	      \end{proof}

	\item Show that Lipschitz continuous functions are continuous. That is, show that if for $f:\RR\to \RR$ there exists a constant $C>0$ such that
	      \[
		      \abs{f(x) - f(y)} \leq C\abs{x-y}
	      \]
	      for all $x,y\in\RR$ then $f$ is continuous.

	      \begin{proof}
		      Let $c$ be an arbitrary point on the real line and let $\epsilon>0$ be given. Define $\delta = \epsilon/C$ and notice that whenever $\abs{x-c} < \delta$ there holds
		      \[
			      \abs{f(x) - f(x)} \leq C\abs{x-c} < C\delta = \epsilon.
		      \]
	      \end{proof}


	\item Suppose $f:\RR\to\RR$ is a continuous function such that $f(q) = 0$ for all $q\in \QQ$. Show that $f\equiv 0$.

	      \begin{proof}
		      Let $x$ be an arbitrary point in $\RR$. For each $n\in\NN$, there exists $x_n \in \QQ$ such that
		      \[
			      x < x_n < x+ \frac{1}{n}.
		      \]
		      By the Squeeze theorem, $x_n \to x$. Since $f$ is continuous, there must hold
		      \[
			      f(x) = \lim_{n\to\infty}f(x_n) = \lim_{n\to\infty} 0 = 0.
		      \]
		      Since $x$ was an arbitrary point in $\RR$, we conclude that $f\equiv 0$.
	      \end{proof}

	\item Let $f:(0,1)\to\RR$ be a bounded function such that $\lim_{x\to 0} f(x)$ does not exist. Show that there are sequence $(x_n)$ and $(y_n)$ converging to $0$ such that $f(x_n)$ and $f(y_n)$ exist but are not equal.

	      \begin{proof}
		      We prove this statement by contrapositive. That is, suppose that there exists $L \in \RR$ such that if $x_n\to 0$ and $\lim_{n\to\infty}f(x_n)$ exists then
		      \[
			      \lim_{n\to\infty} f(x_n) = L.
		      \]
		      We will show that $\lim_{x\to 0} f(x)$ exists. More specifically, we prove that $\lim_{x\to 0}f(x) = L$.

		      Suppose for a contradiction that this is not the case, i.e. $\lim_{x\to 0}f(x)$ does not exist or is not equal to $L$. Then
		      \[
			      (\exists \epsilon > 0)(\forall\delta>0)(\exists x\in (0,1),\, \abs{x}<\delta \text{ such that } \abs{f(x) - L} \geq \epsilon)
		      \]
		      Then for each $n\in \NN$, there exists $x_n$ satisfying $0<x_n<1/n$ and $\abs{f(x_n) - L} \geq \epsilon$.

		      On the other hand, since $f$ is bounded, the Bolzano-Weierstrass theorem implies that $f(x_n)$ has a convergent subsequence $f(x_{n_k})$. By assumption,
		      \[
			      \lim_{n\to\infty} f(x_{n_k}) = L.
		      \]
		      But this contradicts the fact that $\abs{f(x_{n_k}) - L} \geq \epsilon$ for all $k\in\NN$.
	      \end{proof}

	\item Suppose $f, g :\RR\to\RR$ are continuous functions that agree on the rationals. Then is it true that $f \equiv g$?

	      \begin{proof}
		      If $f$ and $g$ are both continuous then so is $f-g$. But then $f-g$ is $0$ on the rationals. By a previous problem, it follows that $f - g\equiv 0$ whence $f\equiv g$.
	      \end{proof}

	\item The above result implies that if we want to show that two functions are equal on all of $\RR$, it suffices to prove that they are equal on the rationals (actually, any dense set will do!). Use this fact to show that if $f:\RR\to\RR$ is continuous and satisfies $f(x+y) = f(x) + f(y)$ for all $x,y\in\RR$ then $f(x) = cx$ where $c = f(1)$.

	      \begin{proof}
		      First, notice that
		      \[
			      f(0) = f(0+0) = f(0) + f(0)
		      \]
		      whence $f(0) = 0 = c\cdot 0$.
		      Now, if $p,q \in \NN$ then
		      \[
			      qf\left(\frac{p}{q}\right) = q\cdot \left[pf\left(\frac{1}{q}\right)\right] = p\cdot\left[qf\left(\frac{1}{q}\right)\right] = pf(1) = p\cdot c
		      \]
		      and dividing by $q$ on either side yields
		      \[
			      f\left(\frac{p}{q}\right) = c\frac{p}{q}.
		      \]
		      This shows that $f(x) = cx$ whenever $x$ is a non-negative rational number. Finally, if $x\in \QQ$ is negative then
		      \[
			      f(x) + f(-x) = f(x+(-x)) = f(0) = 0
		      \]
		      whence
		      \[
			      f(x) = -f(-x)  =-\left(c(-x)\right) = cx
		      \]
		      where the second equality holds since $-x$ is a positive rational number. So, $f(x) = cx$ for all $x\in \QQ$ and since both $f$ and $x\mapsto cx$ are continuous on $\RR$, we conclude that $f(x) = cx$ on all of $\RR$.
	      \end{proof}

	\item Suppose $f:[a,b] \to\RR$ is continuous an strictly positive. Show that there exists $\alpha > 0$ such that $f(x) \geq \alpha$ for all $x\in [a,b]$.

	      \begin{proof}
		      Since $f$ is continuous on a closed and bounded interval, there exists $c\in [a,b]$ such that $f(c)$ is the minimum of $\set{f(x) : x\in [a,b]}$. Therefore, picking $\alpha = f(c)$ concludes the proof.
	      \end{proof}

	\item Suppose $f:\RR\to\RR$ is such that
	      \[
		      \lim_{x\to \infty} f(x) = 0 = \lim_{x\to -\infty} f(x).
	      \]
	      Then show that $f$ achieves a global maximum or a global minimum on $\RR$.

	      \begin{proof}
		      If $f \equiv 0$ then we are done. Otherwise, let $x_0 \in \RR$ be such that $f(x) \neq 0$. We will only consider the case $f(x_0) > 0$ (the case $f(x_0) < 0$ is handled similarly). Since $f(x) \to 0$ as $x \to -\infty$, we may find $a \in \RR$ such that $\abs{f(x)} < f(x_0)$ for all $x < a$. Notice that we must have $a<x_0$. Similarly, we may find $b>x_0$ such that $\abs{f(x)} < f(x_0)$ for all $x > b$.

		      Let now $c$ be such that $f(c)$ is the maximum of $f$ on $[a,b]$. Then $f(c)$ is also the global maximum of $f$ on $\RR$.
	      \end{proof}

	\item Show that the only continuous functions from $\RR$ to $\QQ$ are the constant functions.

	      \begin{proof}
		      Suppose for a contradiction that $f:\RR\to\QQ$ is not constant. Then we may find $a < b$ such that $f(a) \neq f(b)$. By density of the irrationals, we may find a number $y \not\in \QQ$ between $f(a)$ and $f(b)$. Then applying the intermediate value theorem on the restriction of $f$ to $[a,b]$, we see that there exists $c\in [a,b]$ such that $f(c) = y \not\in \QQ$. But this contradicts the assumption that $f$ maps to the rationals.
	      \end{proof}

\end{enumerate}


\end{document}