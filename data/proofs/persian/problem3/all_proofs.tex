\documentclass{article}
\usepackage{amsmath}
\usepackage{xepersian}
\settextfont[Scale=1]{B Nazanin} % Optional font setting

\begin{document}

\title{اثبات اتحاد مثلثاتی}
\author{}
\date{}
\maketitle

\section*{مسئله}
اثبات کنید که:
\begin{equation}
    \left(\frac{1}{\cos a} + \tan a\right)(1 - \sin a) = \cos a.
\end{equation}

\section*{اثبات اول: ساده‌سازی جبری}

ابتدا سمت چپ اتحاد (\lr{LHS}) را گسترش می‌دهیم:
\begin{equation}
    \left(\frac{1}{\cos a} + \frac{\sin a}{\cos a}\right)(1 - \sin a).
\end{equation}

عبارت داخل پرانتز را ساده می‌کنیم:
\begin{equation}
    \frac{1 + \sin a}{\cos a} (1 - \sin a).
\end{equation}

حاصل‌ضرب دو جمله را محاسبه می‌کنیم:
\begin{equation}
    \frac{(1 + \sin a)(1 - \sin a)}{\cos a}.
\end{equation}

با استفاده از اتحاد $(a+b)(a-b) = a^2 - b^2$ داریم:
\begin{equation}
    1 - \sin^2 a = \cos^2 a.
\end{equation}

پس می‌توان نوشت:
\begin{equation}
    \frac{\cos^2 a}{\cos a} = \cos a.
\end{equation}

که برابر با سمت راست اتحاد (\lr{RHS}) است. بنابراین اتحاد اثبات شد.

\section*{اثبات دوم: تبدیل به توابع کسری}

با نوشتن تابع تانژانت به‌صورت کسر داریم:
\begin{equation}
    \tan a = \frac{\sin a}{\cos a} \Rightarrow \frac{1}{\cos a} + \tan a = \frac{1 + \sin a}{\cos a}.
\end{equation}

حال ضرب در $(1 - \sin a)$ را انجام می‌دهیم:
\begin{equation}
    \frac{(1 + \sin a)(1 - \sin a)}{\cos a}.
\end{equation}

چون $1 - \sin^2 a = \cos^2 a$، داریم:
\begin{equation}
    \frac{\cos^2 a}{\cos a} = \cos a.
\end{equation}

پس اتحاد برقرار است.

\end{document}