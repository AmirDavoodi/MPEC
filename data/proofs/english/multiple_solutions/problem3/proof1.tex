\documentclass{article}
\usepackage{amsmath}

\begin{document}

\section{Question:}

Prove that $\sqrt{2}$ is an irrational number.

\section{Answer:}

Assume, for contradiction, that $\sqrt{2}$ is rational. This means it can be written as a fraction of two integers in lowest terms:
\begin{equation}
    \sqrt{2} = \frac{p}{q}, \quad \text{where } p, q \text{ are integers with } \gcd(p, q) = 1.
\end{equation}

Squaring both sides:
\begin{equation}
    2 = \frac{p^2}{q^2}.
\end{equation}

Multiplying both sides by $q^2$:
\begin{equation}
    2q^2 = p^2.
\end{equation}

Since $p^2$ is divisible by $2$, it follows that $p$ must also be divisible by $2$ (as squares of odd numbers are odd). So we write:
\begin{equation}
    p = 2k \quad \text{for some integer } k.
\end{equation}

Substituting into the equation:
\begin{equation}
    2q^2 = (2k)^2 = 4k^2.
\end{equation}

Dividing both sides by $2$:
\begin{equation}
    q^2 = 2k^2.
\end{equation}

This implies that $q^2$ is also divisible by $2$, so $q$ must also be divisible by $2$. This contradicts our original assumption that $p$ and $q$ have no common factors. Therefore, $\sqrt{2}$ is irrational.

\end{document}


